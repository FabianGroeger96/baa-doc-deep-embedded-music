\chapter{Einleitung}
\label{ch:Einleitung}
In diesem Kapitel wird das Projekt eingeführt, so dass klar wird, was mit der Arbeit erreicht werden soll und aus welchem
Grund diese durchgeführt wird.

\section{Aufgabenstellung und Zielsetzung}
\label{sec:Aufgabenstellung-Zielsetzung}
In dieser Sektion wird die Aufgabenstellung sowie die Zielsetzung des Projektes genauer beschrieben. Ziel ist es,
aufzuzeigen, was genau mit dem Projekt erreicht werden möchte und aus welcher Situation dieses Projekt entstanden ist.
Die ausführliche Aufgabenstellung, welche bei der Transferstelle der Hochschule Luzern eingereicht wurde, ist im Anhang
\fullref{ch:aufgabenstellung} zu finden.

\subsection{Ausgangslage}
\label{sub:ausgangslage}
Mit dem Zeugnismanager der confer! AG können Arbeits- und Zwischenzeugnisse in vier Sprachen erstellt werden. Die
Zeugnisse werden durch einfaches anwählen von Bausteinen zusammengestellt. Die so entstehenden Zeugnisse sind «holprig»
zu lesen. Ein Beispiel ist der Einsatz der Anrede (Frau Meier) und des Personalpronomens (Sie) an geeigneter Stelle.
Solche und ähnliche Problemstellungen werden im Moment mit einfachen Heuristiken gelöst. Weitere Probleme ergeben sich
z. B. durch die unterschiedlichen Grammatiken in den jeweiligen Sprachen. So müssen viele Spezialfälle behandelt werden.

\subsection{Aufgabenstellung}
\label{sub:aufgabenstellung}
Das Ziel ist es, den Zeugnismanager dabei zu unterstützen, sprachlich korrekt ausformulierte Zeugnisse mit einem guten
Lesefluss zu erstellen. Es soll eine Komponente mit Prototypcharakter erarbeitet werden. Der zu erarbeitende Prototyp
kann sich auf eine Sprache, Zeitform und ggf. sogar Geschlecht einschränken, sollte aber entsprechend erweiterbar sein.

\subsection{Zielsetzung}
\label{sub:zeilsetzung}

Wie in der Aufgabenstellung im Anhang \fullref{ch:aufgabenstellung} beschrieben wird, ist das das Projekt ausserhalb des
Tagesgeschäft angesiedelt und Teil von Forschung und Entwicklung. Daher wird das vorliegende Projekt nach einem
explorativen Ansatz durchgeführt. Aus verschiedenen Lösungsansätzen soll ein Ansatz überprüft werden und eine Empfehlung
für den Einsatz beim Arbeitgeber abgegeben werden. Dadurch wird die Zielsetzung und Formulierung der genauen
Problemdefinition im Verlauf der Arbeit vorgenommen. Da es sich um ein exploratives Innovations- / Forschungsprojekt
ist die Erreichung der Aufgabenstellung und Zielsetzung zu Beginn schwer einschätzbar.