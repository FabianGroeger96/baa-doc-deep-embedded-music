\chapter{Fazit}
\label{ch:Fazit}

Zum Abschluss der Arbeit soll für den Auftaggeber ein Fazit gezogen, sowie ein Ausblick gemacht werden. Diese können als
Empfehlungen für die weitere Verwendung der Erkenntnisse der Arbeit verwendet werden.

\section{Projekt Fazit}
\label{projekt_fazit}
Bereits zu Beginn des Projekts wurde eine Problematik aufgezeigt. Für das erfolgreiche Nutzen von Ansätzen aus
maschinellem Lernen und künstlicher Intelligenz, sind umfangreiche Datensätze vorausgesetzt. Diese sind im expliziten
Bereich von Zwischen- und Arbeitszeugnissen nicht vorhanden. Auch fanden sich keine vergleichbaren Daten, welche eine
Anwendung in der produktiven Umgebung des Arbeitgebers verwenden liessen. Zu Beginn des Projektes wurden die möglichen
Ansätzen zur Verbesserung der generierten Arbeitszeugnissen durchgespielt, siehe auch
\fullref{sec:empfehlung_zeugnismanager}. Dabei muss aufgeführt werden, dass der Verwendung von \gls{NST} mit
neuronalen Netzwerken bereits zu diesem Zeitpunkt, in dem kurzen Projektzeitraum, keine grosse Erfolgschance
zugesprochen wurde. Weiterentwicklungen des Zeugnismanager im Bereich des User Interfaces, des Bewertungsprozesses 
oder der manuellen Verbesserung der Textbausteine, wurden höhere Erfolgschancen eingeräumt. 
\newline
\newline
Im Rahmen des Projektantrags für das Wirtschaftsporjekt an der HSLU, sowie in Absprache mit der Betreuungsperson, sollte
das Projekt jedoch klar im Bereich von maschinellem Lernen sowie künstlicher Intelligenz angesiedelt sein. Daher wurde
entschieden den Style Transfer für die Verwendung im Zeugnismanager zu untersuchen. Dies auch da es sich bei diesem
Wirtschaftsprojekt um ein Innovationsprojekt handelt. Auch hat der Auftraggeber bisher keine Berührungspunkte mit
maschinellem Lernen oder künstlicher Intelligenz. Daher soll die Arbeit auch einen ersten Einblick in die Disziplin
gewähren.
\newline
\newline
Durch das junge Alter von \gls{NST} und der geringen Anzahl an kommerziell eingesetzten Lösungen, gestaltete
es sich schwierig die entsprechenden Ansätze für die Aufgabenstellung einzuschätzen. Anhand des Sammelsuriums an
Literatur aus der Forschung, siehe \cite{fuzhenxin_2019}, lässt sich sehen, dass \gls{NST} in Kombination mit \gls{NLP}
gerade eine starke Entwicklung durchläuft. Ein genaues Sichten dieser Arbeiten, sowie das Einschätzen der Resultate
stellt bereits ein hoher Aufwand dar. 
\newline
\newline
Weiter war zum Start der Arbeit nicht klar, welcher Datensatz verwendet werden soll. Durch die Entscheidung einen
eigenen Datensatz, welcher das Problem annähern soll, zu anzulegen, wurde ein Mehraufwand generiert. Dies war sicherlich
nötig um eine Untersuchung zu ermöglichen, beanspruchte jedoch einen grossen Teil der Zeitressourcen.
\newline
\newline
Die Resultate lassen durchaus darauf schliessen, dass \gls{NST} eine Möglichkeit zur Verbesserung von Sätzen wäre. Auch
da, dass Ziel Sätze zu verlängern und zu verkürzen kann teilweise als erreicht gewertet werden kann. Dennoch sind die
Resultate in keinster Weise mit den Resultate aus der Literatur vergleichbar. Die Gründe warum nicht ähnliche Resultate
erreicht werden konnte können schwer erörtert werden. Das Projekt behandelte eine grosse Anzahl verschiedener Aspekte
aus maschinellem Lernen und künstlicher Intelligenz. Einerseits wurde der Datensatz eigenhändig aufgebaut, sowie der
Transfer auf Problem angewandt welches sich von den Problemstellung in de Literatur unterscheidet. 
\newline
\newline
Weiter soll nochmals Erwähnt werden, dass es sich beim Projekt um ein exploratives Projekt handelt. Es ging darum
Versuche und Möglichkeiten die Aufgabenstellung mit künstlicher Intelligenz zu lösen. Dabei ist ein Scheitern leider
nicht auszuschliessen.

\section{Empfehlung im Bezug auf Neural Style Transfer}
\label{sec:empfehlung_nst}

\gls{NST} kann im Bereich von \gls{NLG} und \gls{NLP} durchaus Erfolge verzeichnen und wird zukünftig eine Rolle
einnehmen. Die Forschung in diesem Bereich verspricht ein grosses Potenzial und wird sich in den nächsten Jahren stark
wandeln. Es ist durchaus möglich, dass Ansätze und Modelle auftauchen werden die den gewünschten Transfer durchführen
können. Dabei darf jedoch nicht vernachlässigt werden, dass ein umfangreicher Datensatz dafür notwendig ist. Auch wenn
es Ansätze gibt Themenbereich übergreifend Transfers durchzuführen, bleibt das Problem der Dimensionalität in \gls{NLP}
und \gls{NLG} wohl weiterhin bestehen. Um eine deutliche Verbesserung der Sätze mit dem Transfer zu erhalten, muss ein
detailierter Raum für die Bedeutung einzelner Wörter geschaffen werden. Gerade in einem Bereich, wie in
persönlich-adressierten Zeugnissen, wird eine hohe Detailtreue gefordert. Daher ist ein grosser und geprüfter Datensatz
aus Arbeitszeugnissen unumgänglich.
\newline
\newline
Abschliessend werden für den Auftraggeber im Bezug auf Neural Style Transfer und dem Zeugnismanager folgende Empfehlung
abgegeben.

\begin{itemize}
    \setlength\itemsep{0em}
  \item Weiterführen der Arbeit um Ergebnisse für eine abschliessende Beurteilung zu erhalten
  \item Untersuchung von alternativ Modellen aus dem Bereich von \gls{NST}
  \item Wiederaufnahme der Thematik in drei bis fünf Jahren
  \item Aufbau eines Datensatz welcher den Style Transfer im Themenbereich von persönlich-adressierten Zwischen- und
  Arbeitszeugnissen abbildet
\end{itemize}

\section{Empfehlung im Bezug auf den Zeugnismanager}
\label{sec:empfehlung_zeugnismanager}

Dem Auftraggeber sollen ebenfalls Ansätze zur Verbesserung des Zeugnismanager ausserhalb von \gls{NST} vorgeschlagen
werden. Auch wenn der Zeugnismanager an sich nicht untersucht wurde, können gewisse Vorschläge eingebracht werden.

\paragraph{Verbesserung des User Interface durch Abstimmung} Anstelle die Zeugnisse mit Textbausteinen für einzelne
Kategorien zu erstellen, könnte für die einzelnen Kategorien über das Wählen von Adjektiven ein Zeugnis erstellt werden.
Dabei könnte zum Beispiel eine Art Abstimmung zwischen zwei Adjektiven durchgeführt werden. Dabei wäre es möglich für
eine Kategorie zwei bis drei Abstimmungen durchzuführen. Anschliessend kann Anhand dessen ein Textbaustein oder auch ein
vollständiges Zeugnis vorgeschlagen werden. Dadurch könnte der Verfasser des Zeugnis sich auf die genaue Bewertung
fokussieren.

\paragraph{Einsatz von Fuzzy Logic} Mit Fuzzy Logic (\cite{fuzzy_logic}) wird es möglich Wörter für eine Kategorie zu
bewerten. Anstelle Wörter Binär in beispielsweise \flqq positiv\frqq \ und \flqq negativ\frqq \ einzuteilen, wird bei
der Fuzzy Logic ein Wort im Interval $ [0,1] $ eingeteilt. Dadurch wird es möglich Wörter wie beispielsweise \flqq
gut\frqq, \flqq hervorragend\frqq \ und \flqq herrausragend\frqq \ genauer zu gewichten.
\newline
\newline
Im Zeugnismanager könnte nun ein Score über das mit den Textbausteinen erzeugte Zeugnis erstellt werden. Dieser Score
kann über die verschienden Abschnitte erstellt werden. Anschliessend könnte die Kohärenz der Scores überprüft werden und
so Verbesserungsvorschläge erstellt werden. Auch wäre es möglich mit einem Slider dem Verfasser ein Feineinstellung der
relevanten Wörter in einem Textbausteine zu ermöglichen.
