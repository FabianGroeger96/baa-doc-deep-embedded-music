\chapter{Milestone reports}
\label{app:Milestone-Reports}
In diesem Kapitel werden die einzelnen Berichte der Meilensteine aufgeführt. Diese sollen Aufschluss über den Stand des
Projektes während der Umsetzung geben.

\section{Meilensteinbericht M1 vom 26. September 2019}

\subsection{Aktueller Stand bei Meilenstein}

\subsection{Ausblick zum nächsten Meilenstein}

\chapter{Research}
\label{app:Research}

\section{Links}
\begin{itemize}
    \item List of various useful links: <https://github.com/fuzhenxin/Style-Transfer-in-Text>
\end{itemize}

\section{Papers}
All paper listed. Every paper will assigned a number to make communication easier. Papers are tried to be hold in
categories.

\subsection{Overview - Neural Style Transfer}
Papers which describe neural style transfer in general and show a general overview over the existing technologies and
ideas.

\subsubsection{01 - What is wrong with style transfer for texts?}
\textbf{Link: } <https://arxiv.org/pdf/1808.04365.pdf>
\newline
\newline
Das Paper bietet eine gute Übersicht über die aktuelle Lage mit dem Thema Neural Style Transfer. Die Übersicht wird
unter der Frage nach den schwächen des Bereichs.
\newline
\newline
Dabei wird eine Übersicht über die vorhanden vorgeschlagenen Modelle gegeben.
\begin{itemize}
    \item \textbf{Ad-hoc defined:} Der Style Transfer erfolgt über ein Datenset eines Autoren oder Platform. Dabei wird
    versucht der Stil des Datenset wiederzugeben. Dabei wird, z.B. mit einem LSTM, ein neuer Text aus dem Set generiert.
    Es handelt sich somit um die generierung von neuem Text aufgrund eines Datensatz.
    \item \textbf{Neural Machine Translation (NMT):} Die Stile werden als Sprachen angesehen. Dabei wird ein Übersetzer
    für den einen in den anderen Stil (Sprache) verwendet. So mit kann mithilfe eines parallen Datensatzes stilisiert
    werden.
    \item \textbf{Post NMT:} Allgemein wurde sich geeinigt das der Style Transfer nicht aufgrund von Fragementen
    getätigt werden soll. Eher soll der Transfer aufgrund der Daten parametrisiert werden. Dieser Ansatz wird momentan
    stark weiterverfolgt und erweitert.
\end{itemize}
Weiter wird im Paper die Problemstellung der Definition für Text-Semantik und Text-Stil aufgeführt.

\chapter{Work journal}
\label{ch:Work-Journal}

\landscapevalues

\begin{table}[ht]
	\centering
	\caption{Work Journal}
	\label{tab:Work-Journal}
	\begin{tabular}{|l|l|l|l|}
    \hline
    \textbf{Date} &
    \textbf{Task} &
    \textbf{Details} &
    \textbf{No. hours} \\
    \hline
    \multicolumn{4}{|l|}{\textbf{General}} \\
    \hline
    17.02.2020 & Preparation for Kick-Off Meeting & 
        \begin{minipage}{3in}
        \vskip 4pt
        \begin{itemize}
        \setlength\itemsep{0em}
        \item thesis description read carefully again
        \item questions of unclear matters
        \end{itemize}
        \vskip 4pt
        \end{minipage}
        & 1h  \\
    \hline
    17.02.2020 & Kick-Off Meeting & 
        \begin{minipage}{3in}
        \vskip 4pt
        \begin{itemize}
        \setlength\itemsep{0em}
        \item discussed different topics which are important for the implementation of the project
        \end{itemize}
        \vskip 4pt
        \end{minipage}
        & 1h 30min  \\
    \hline
    \multicolumn{4}{|l|}{\textbf{Documentation}} \\
    \hline
    17.02.2020 & Documentation setup & 
        \begin{minipage}{3in}
        \vskip 4pt
        \begin{itemize}
        \setlength\itemsep{0em}
        \item project setup
        \item latex document setup
        \item German transcribed to English
        \item built default documentation structure
        \end{itemize}
        \vskip 4pt
        \end{minipage}
        & 2h  \\
    \hline
    \multicolumn{4}{|l|}{\textbf{Research}} \\
    \hline
    17.02.2020 & Research Dataset & 
        \begin{minipage}{3in}
        \vskip 4pt
        \begin{itemize}
        \setlength\itemsep{0em}
        \item finished with DCASE - Challenge description
        \item started with SINS Database Paper
        \end{itemize}
        \vskip 4pt
        \end{minipage}
        & 2h 30min  \\
    \hline
    \end{tabular}
\end{table}

\clearpage
\defaultpagestyle
