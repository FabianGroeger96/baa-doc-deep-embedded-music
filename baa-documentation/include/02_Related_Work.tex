\chapter{Related Work}
\label{ch:Related-Work}

This chapter introduces and explains terms and concepts which will be used throughout the thesis. Thus, this section can be used as a reference guide to fill in any gaps and to understand the relationships between the individual concepts better.

\section{Technological fundamentals}
\label{sec:Technological-Fundamentatls}

This section deals with the essential technological fundamentals of the project. These basics contain mainly definitions of terms, which will be used repeatedly in the following chapters. Most of these principles are superordinate terms, which are required to understand the detailed concepts in \fullref{sec:Technological-Fundamentatls}.

\subsection{Tensorflow}
\label{sub:Tensorflow}

\section{Technical concepts}
\label{sec:Technical-Concepts}

In this section, technical concepts are explained in more detail, which was used in the thesis. These concepts are mostly very complex and are, therefore only touched upon so that the further conclusions of this work can be understood comprehensibly.

\subsection{Word2Vec}
\label{sub:word2wec}

\subsection{Triplet Loss}
\label{sub:Triplet-Loss}

\subsection{Mel Spectogram}
\label{sub:Mel-Spectogram}

\subsection{Fourier Transformation}
\label{sub:Fourier-Transformation}

\subsubsection{Fast Fourier Transformation}
\label{subsub:Fast-Fourier-Transformation}

\subsubsection{Short Time Fourier Transformation}
\label{subsub:Short-Time-Fourier-Transformation}

\subsection{$\mu$-law Transformation}
\label{sub:Mu-Law-Transformation}

\section{Status in relation to project}
\label{sec:Status-Relation-Project}
