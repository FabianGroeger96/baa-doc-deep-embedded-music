\documentclass[a4paper]{scrreprt}

%%% PACKAGES %%%

%etoolbox for rules
\usepackage{etoolbox}

% Figure Placement
\usepackage{float}
% PDF/A Compliance
%\usepackage[a-2b,latxmp]{pdfx}

% add Unicode support and use English as language
\usepackage[utf8]{inputenc}
\usepackage[english]{babel}

% Use Times NR as font
\usepackage{lmodern}
\usepackage[T1]{fontenc}

% Better tables
\usepackage{tabularx}

% Better enumerisation env
\usepackage{enumitem}
\usepackage{multirow}

% Use graphics
\usepackage{graphicx}

% Have subfigures and captions
\usepackage{subcaption}
\usepackage{caption}

% Be able to include PDFs in the file
\usepackage{pdfpages}

% Have custom abstract heading
\usepackage{abstract}

% Need a list of equation
\usepackage{tocloft}
\usepackage{ragged2e}

\usepackage{dirtree}
\usepackage{adjustbox}

% Better equation environment
\usepackage{amsmath}

% Symbols for most SI units
\usepackage{siunitx}

\usepackage{csquotes}

% to make beautiful tables
\usepackage{booktabs}

% to syntax highlighted code samples
\usepackage{listings}

% Clickable Links to Websites and chapters
\usepackage[hidelinks]{hyperref}

% to have multiple footnotes at the same point in the text
\usepackage[multiple]{footmisc}

% New command for 'fullref'
\newcommand*{\fullref}[1]{\hyperref[{#1}]{\nameref*{#1} (\ref*{#1})}}

% lscape.sty Produce landscape pages in a (mainly) portrait document.
%\usepackage{lscape}
\usepackage{pdflscape}

% Symbols like checkmark
\usepackage{amssymb}
\usepackage{pifont}

% Lorem ipsum package for default text
\usepackage{lipsum}

\usepackage[absolute]{textpos}

\usepackage{longtable}

% Header
\usepackage{scrlayer-scrpage}

\clearpairofpagestyles

% sets the font of the document headers to Times NR
\addtokomafont{disposition}{\rmfamily}

 % Configures header and footer
\setkomafont{pageheadfoot}{\rmfamily\footnotesize}
\setkomafont{pagehead}{\rmfamily\bfseries}
\setkomafont{pagination}{}

% Displays line for header and footer
\KOMAoptions{
   headsepline = true,
   footsepline = true,
   plainfootsepline = true,
}

% Tikz to crate diagrams, thanks to: https://github.com/mvoelk/nn_graphics
% Start of tikz settings
\usepackage{tikz}
\usetikzlibrary{positioning,arrows.meta}
\usetikzlibrary{matrix, chains, positioning, decorations.pathreplacing, arrows}
\usetikzlibrary{shapes,arrows,positioning,calc,chains,scopes}

\usepackage{ifthen}
\usepackage{pgfplots}
\pgfplotsset{compat=1.16}
\pgfplotsset{every axis/.append style={tick label style={/pgf/number format/fixed},font=\scriptsize,ylabel near ticks,xlabel near ticks,grid=major}}

\usepackage{amsmath}
\DeclareMathOperator{\sigm}{sigm}
\newcommand{\diff}{\mathop{}\!\mathrm{d}}

% colors
\definecolor{snowymint}{HTML}{E3F8D1}
\definecolor{wepeep}{HTML}{FAD2D2}
\definecolor{portafino}{HTML}{F5EE9D}
\definecolor{plum}{HTML}{DCACEF}
\definecolor{sail}{HTML}{A3CEEE}
\definecolor{highland}{HTML}{6D885A}

\tikzstyle{signal}=[arrows={-latex},draw=black,line width=1pt,rounded corners=4pt]

% RNN
\tikzstyle{block}=[draw=black,line width=1.0pt]
\tikzstyle{cell}=[style=block,draw=highland,fill=snowymint,
    rounded corners]
\tikzstyle{celllayer}=[style=block,draw,fill=portafino,
    inner sep=1pt,outer sep=0,
    minimum width=28pt, minimum height=14pt]
\tikzstyle{pointwise}=[style=block,ellipse,fill=wepeep,
    inner sep=1pt,outer sep=0, minimum size=12pt]

\def\iolen{24pt}
\def\intergape{2pt}

% MLP and CNN
\tikzstyle{netnode}=[circle, inner sep=0pt, text width=22pt, align=center, line width=1.0pt]
\tikzstyle{inputnode}=[netnode, fill=sail, draw=black]
\tikzstyle{hiddennode}=[netnode, fill=snowymint, draw=black]
\tikzstyle{infonode}=[netnode, fill=portafino, draw=black, inner sep=6pt, font=\Large]
\tikzstyle{outputnode}=[netnode, fill=plum, draw=black]

% Architecture
\def\layerwidth{120pt}
\def\layerheight{30pt}

\tikzstyle{layer}=[style=block, draw, fill=black!20!white,
    inner sep=5pt,outer sep=0pt, font=\footnotesize,
    text centered, align=center,
    minimum width=\layerwidth, minimum height=\layerheight]

\tikzstyle{fc}=[style=layer, fill=blue!30!white]
\tikzstyle{conv}=[style=layer, fill=green!30!white]
\tikzstyle{activation}=[style=layer, fill=orange!30!white]
\tikzstyle{pool}=[style=layer, fill=red!30!white]
\tikzstyle{bn}=[style=layer, fill=cyan!30!white]
\tikzstyle{recurrent}=[style=layer, fill=purple!30!white]
\tikzstyle{softmax}=[style=layer, fill=yellow!30!white]
\tikzstyle{point}=[]
\tikzstyle{branch}=[coordinate]

\def\vlayerwidth{30pt}
\def\vlayerheight{3pt}
\def\vblockheight{28pt}

\tikzstyle{vlayer}=[minimum width=\vlayerwidth, minimum height=\vlayerheight]
\tikzstyle{vblock}=[minimum width=\vlayerwidth, minimum height=\vblockheight, text width=1cm, align=center]


% Precision, Recall
\colorlet{fn}{gray!90!green!30!white}
\colorlet{tp}{green!40!white}
\colorlet{fp}{red!40!white}
\colorlet{tn}{gray!90!red!20!white}

% End of tikz settings

\usepackage{pgfgantt}
\usepackage{graphicx}
\usepackage{xcolor}

\ganttset{group/.append style={orange},
milestone/.append style={red},
progress label node anchor/.append style={text=red}}

\automark[chapter]{chapter}

\ohead{\headmark}
\ihead{Fabian Gröger}

\ofoot*{\pagemark}
\ifoot*{HSLU BAA - Deep embedded Music}

% Glossary, hyperref, babel, polyglossia, inputenc, fontenc must be loaded before this package if they are used
\usepackage[ 
xindy,
toc,         
acronym,
nopostdot,
sort=standard   
]{glossaries}

\setglossarystyle{listgroup}

% Redefine the quote charachter as we are using ngerman
%\GlsSetQuote{+}
% Define the usage of an acronym, Abbreviation (Abbr.), next usage: The Abbr. of ...
\setacronymstyle{long-short}

% Bibliography & citing
\usepackage[
	backend=biber,
	style=authoryear,
	bibstyle=authoryear,
	citestyle=authoryear
	]{biblatex}
\addbibresource{references_zotero.bib}
\addbibresource{Referenzen.bib}
\addbibresource{Links.bib}
\DeclareLanguageMapping{english}{english-apa}

%%% COMMAND REBINDINGS %%%
\newcommand{\tabitem}{~~\llap{\textbullet}~~}
\newcommand{\xmark}{\ding{55}}
\newcommand{\notmark}{\textbf{\textasciitilde}}
% Pro/Con item https://tex.stackexchange.com/questions/145198/change-the-bullet-of-each-item#145203
\newcommand\pro{\item[$+$]}
\newcommand\con{\item[$-$]}

% Define list of equations - Thanks to Charles Clayton: https://tex.stackexchange.com/a/354096
\newcommand{\listequationsname}{List of Equations}
\newlistof{myequations}{equ}{\listequationsname}
\newcommand{\myequations}[1]{
	\addcontentsline{equ}{myequations}{\protect\numberline{\theequation}#1}
}
\setlength{\cftmyequationsnumwidth}{2.3em}
\setlength{\cftmyequationsindent}{1.5em}

% Usage {equation}{caption}{label}
\newcommand{\indexequation}[3]{
	\begin{align} \label{#3} \ensuremath{#1} \end{align}
	\myequations{#3} \centering #2 \normalsize \justify }
	
% equations with term definitions, credit to tex.stackexchange.com/questions/95838/how-to-write-a-perfect-equation-parameters-description
\newenvironment{conditions}
  {\par\vspace{\abovedisplayskip}\noindent\begin{tabular}{>{$}l<{$} @{${}={}$} l}}
  {\end{tabular}\par\vspace{\belowdisplayskip}}
  
\newenvironment{conditions*}
  {\par\vspace{\abovedisplayskip}\noindent
   \tabularx{\columnwidth}{>{$}l<{$} @{${}={}$} >{\raggedright\arraybackslash}X}}
  {\endtabularx\par\vspace{\belowdisplayskip}}

% Todolist - credit to https://tex.stackexchange.com/questions/247681/how-to-create-checkbox-todo-list
\newlist{todolist}{itemize}{1}
\setlist[todolist]{label=$\square$}

% Nested Enumeratelist credit to https://tex.stackexchange.com/a/54676
\newlist{legal}{enumerate}{10}
\setlist[legal]{label*=\arabic*.}

% Formatting Chapter title, thanks to: 
% http://texnorte.blogspot.com/2012/06/insert-code-before-chapter-title.html
\newcommand\toptitle{}
\makeatletter
\def\@makechapterhead#1{%
  \vspace*{50\p@}%
  {\parindent \z@ \raggedright \normalfont
    \ifnum \c@secnumdepth >\m@ne
      \if@mainmatter
        \toptitle\par
        \huge\bfseries \@chapapp\space \thechapter
        \par\nobreak
        \vskip 20\p@
      \fi
    \fi
    \interlinepenalty\@M
    \Huge \bfseries #1\par\nobreak
    \vskip 40\p@
  }}
\makeatother

%%% GLOSSARY %%%
%%% ACRONYM ENTRIES %%%
\newacronym{DSP}{DSP}{Digital Signal Processing}
\newacronym{ADC}{ADC}{Analog-to-Digital converter}
\newacronym{DAC}{DAC}{Digital-to-Analog converter}
\newacronym{ASR}{ASR}{Automatic Speech Recognition}
\newacronym{ML}{ML}{Machine Learning}
\newacronym{NN}{NN}{Neuronales Netz}
\newacronym{KNN}{KNN}{Künstliches neuronales Netz}
\newacronym{ANN}{ANN}{Artificial Neuronal Net}
\newacronym{CNN}{CNN}{Convolutional Neural Network}
\newacronym{RNN}{RNN}{Recurrent neural network}
\newacronym{LSTM}{LSTM}{Long short term memory}
\newacronym{GAN}{GAN}{Generative Adversarial Networks}
\newacronym{VAE}{VAE}{Variational Auto-Encoder}
\newacronym{RAM}{RAM}{Random Access Memory}
\newacronym{TPU}{TPU}{Tensor processing unit}
\newacronym{GPU}{GPU}{Graphics processing unit}
\newacronym{Hz}{Hz}{Hertz}
\newacronym{kHz}{kHz}{Kilohertz}
\newacronym{SINS}{SINS}{Sound Interfacing through the Swarm}
\newacronym{DCASE}{DCASE}{Detection and Classification of Acoustic Scenes and Events}
\newacronym{AAL}{AAL}{Ambient Assisted Living}
\newacronym{MFCC}{MFCC}{Mel Frequency Cepstral Coefficients}
\newacronym{FT}{FT}{Fourier Transform}
\newacronym{FFT}{FFT}{Fast Fourier Transformation}
\newacronym{DFT}{DFT}{Discrete Fourier Transformation}
\newacronym{DCT}{DCT}{Discrete Cosine Transformation}
\newacronym{STFT}{STFT}{Short Time Fourier Transform}
\newacronym{NLP}{NLP}{Natural Language Processing}
%%% GLOSSARY ENTRIES %%%
\newglossaryentry{Epoche}{name=Epoche, description={Kompletter Durchlauf aller Input-Daten}}
\newglossaryentry{Batch}{name=Batch, description={Anzahl der Input-Daten bevor die Parameter angepasst werden}}
\newglossaryentry{controlgen}{name=ControlGen, description={Das ControlGen Style Transfer Modell}}
\newglossaryentry{crossalign}{name=CrossAlign, description={Das CrossAlign Style Transfer Modell}}

\makeglossaries

%%% PATH DEFINITIONS %%%
% Define the path were images are found
\graphicspath{{./img/}{./appendix/}}

% Reset Glossary every chapter
\preto\chapter{\glsresetall}

%%%%%%%%%%%%%%%%
%%% DOCUMENT %%%

\begin{document}

% HSLU Preamble (no header or footer)
\pagestyle{empty}

% Cover Page
\begin{titlepage}
	\begin{textblock*}{5cm}[0,0](15.1cm,0.7cm)
		\includegraphics[keepaspectratio,width=5cm]{img/HSLU_Logo}
	\end{textblock*}
	\begin{center}
		\vspace*{5cm}
		\Huge{\textbf{Deep Embedded Music}} \\
		\vspace{0.5em}
		\LARGE{\textsc{Bachelor Thesis}}\\
		\vspace{2em}
		\Large{presented by}\\
		\LARGE{\textsc{Fabian Gröger}}\\
		\vspace{2em}
		\Large{Departement Information Technology}\\
		\Large{Lucerne University of Applied Sciences and Arts}\\
		\Large{6343 Rotkreuz, Switzerland}\\
		\vspace{4em}
		\Large{\today}\\
		\vfill
		\begin{center}
		\resizebox{\textwidth}{!}{%
        \begin{tabular}{ l l }
         Advisor: & Daniel Pfäffli \\ 
         \ & Lucerne University of Applied Sciences and Arts, 6343 Rotkreuz, Switzerland \\
         \ & daniel.pfaeffli@hslu.ch \\ 
         Expert: & TBD
        \end{tabular}}
        \end{center}
	\end{center}
	\begin{textblock*}{5cm}[0,0](15.3cm,277mm)
		\includegraphics[keepaspectratio,width=5cm]{img/FHZ_Logo}
	\end{textblock*}
\end{titlepage}

\newpage
\pagenumbering{gobble}

\begin{textblock*}{5cm}[0,0](15cm,0.7cm)
	\includegraphics[keepaspectratio,width=2.7cm]{img/HSLU_Logo_Header}
\end{textblock*}

\vspace{0.6cm}
\noindent
\textbf{\Large{Bachelorarbeit an der Hochschule Luzern – Informatik}}

\vspace{0.6cm}
\noindent
\textbf{Titel:} Deep Embedded Music

\vspace{0.6cm}
\noindent
\textbf{Studentin / Student:} Fabian Gröger

\vspace{0.6cm}
\noindent
\textbf{Studiengang:} BSc Informatik

\vspace{0.6cm}
\noindent
\textbf{Abschlussjahr:} 2020

\vspace{0.6cm}
\noindent
\textbf{Betreuungsperson:} Daniel Pfäffli, HSLU Informatik, Rotkreuz

\vspace{0.6cm}
\noindent
\textbf{Expertin / Experte:} Dr. Jeremy Callner, APG/SGA, Zürich

\vspace{0.6cm}
\noindent
\textbf{Auftraggeberin / Auftraggeber:} Hochschule Luzern - Informatik

\vspace*{1.00cm}

\noindent
\textbf{Codierung / Klassifizierung der Arbeit:} Innovationsprojekte (Projekte mit
Erkenntnisgewinn, Forschungsprojekte)

\begin{todolist}
	\item \textbf{A: Einsicht (Normalfall)}
	\item \textbf{B: Rücksprache}\hspace*{0.7cm}(Dauer:\hspace*{1cm} Jahr / Jahre)
	\item \textbf{C: Sperre}\hspace*{1.865cm}(Dauer:\hspace*{1cm} Jahr / Jahre)
\end{todolist}

\vspace*{1.00cm}

\noindent
\textbf{Eidesstattliche Erklärung}
\\
Ich erkl\"are hiermit, dass ich/wir die vorliegende Arbeit selbst\"andig und ohne unerlaubte fremde Hilfe angefertigt haben, alle verwendeten Quellen, Literatur und andere Hilfsmittel angegeben haben, w\"ortlich oder inhaltlich entnommene Stellen als solche kenntlich gemacht haben, das Vertraulichkeitsinteresse des Auftraggebers wahren und die Urheberrechtsbestimmungen der Fachhochschule Zentralschweiz (siehe Merkblatt <<Studentische Arbeiten>> auf MyCampus) respektieren werden.

\vspace{2em}

\noindent
\begin{tabularx}{\textwidth}{@{}lX}
	&\\
	Ort / Datum, Unterschrift: &  \\
	\cline{2-2}
\end{tabularx}

\begin{textblock*}{5cm}[0,0](14.93cm,277mm)
	\includegraphics[keepaspectratio,width=5cm]{img/FHZ_Logo}
\end{textblock*}

\newpage

\begin{textblock*}{5cm}[0,0](15cm,0.7cm)
	\includegraphics[keepaspectratio,width=2.7cm]{img/HSLU_Logo_Header}
\end{textblock*}

\noindent
\textbf{Abgabe der Arbeit auf der Portfolio Datenbank:}\\
Best\"atigungsvisum Studentin/Student\\
Ich best\"atige, dass ich die Bachelorarbeit korrekt gem\"ass Merkblatt auf der Portfolio Datenbank abgelegt habe. Die Verantwortlichkeit sowie die Berechtigungen habe ich abgegeben, so dass ich keine \"Anderungen mehr vornehmen kann oder weitere Dateien hochladen kann. \newline \newline 
Ort / Datum, Unterschrift	\underline{\hspace*{10cm}} \newline \newline \newline

\noindent
\textbf{Verdankung}\\
An dieser Stelle möchte ich mich recht herzlich bei meinem Betreuer, Daniel Pfäffli, bedanken, der mich während der ganzen Durchführung unterstützt und motiviert hat. Ich habe von ihm und seinem grossen Wissen enorm profitiert und wurde dadurch angetrieben mehr zu leisten. Ausserdem möchte ich mich herzlich bei Emanuel Oehri bedanken, der sich Zeit genommen hat, das Musik Data Set bereitzustellen und mit mir die qualitative Analyse durchzuführen. Weiter möchte ich mich bei meiner Frau Deborah Gröger bedanken für die Unterstützung über die ganze Dauer der Arbeit. Meinen Eltern Dr. Ulrich Gröger und Madeleine Gröger möchte ich meinen grössten Dank aussprechen für die konstante Unterstützung über die ganze Studienzeit.
\newline \newline \newline

\vspace{0.8cm}
\noindent
\textbf{Ausschliesslich bei Abgabe in gedruckter Form: \\ 
Eingangsvisum durch das Sekretariat auszufüllen}

\noindent
\renewcommand{\arraystretch}{2}
\begin{tabularx}{\textwidth}{@{}lXlX}
	Rotkreuz, den & & Visum: & \\
	\cline{2-2}
	\cline{4-4}
\end{tabularx}
\renewcommand{\arraystretch}{1}

\vfill
\noindent
{\textbf{Hinweis}}: Die Bachelorarbeit wurde von keinem Dozierenden nachbearbeitet. Ver\"offentlichungen (auch auszugsweise) sind ohne das Einverst\"andnis der Studiengangleitung der Hochschule Luzern -- Informatik nicht erlaubt. \newline \newline
Copyright \textcopyright\ {2020} Hochschule Luzern -- Informatik \newline \newline
Alle Rechte vorbehalten. Kein Teil dieser Arbeit darf ohne die schriftliche Genehmigung der Studiengangleitung der Hochschule Luzern -- Informatik in irgendeiner Form reproduziert oder in eine von Maschinen verwendete Sprache \"ubertragen werden.


\begin{textblock*}{5cm}[0,0](14.93cm,277mm)
	\includegraphics[keepaspectratio,width=5cm]{img/FHZ_Logo}
\end{textblock*}


% Rest of the document has header and footer
\pagestyle{headings}
\pagenumbering{Roman}

% Abstract
\begin{abstract}
	\noindent
	\lipsum[50]
\end{abstract}

% Table of contents
\tableofcontents

\clearpage
\pagenumbering{arabic}

\chapter{Introduction}
\label{ch:Introduction}
This chapter aims to introduce the thesis \flqq Deep Embedded Music\frqq \ and describes the task and the objectives of the project in detail. It should give an insight into the situation it arose from, as well as the motivation behind it. The detailed task, which was submitted to the Transfer Office of the Lucerne University of Applied Sciences and Arts, can be found in the appendix \fullref{app:Task-Definition}.

\section{Initial situation}
\label{sub:Initial-Stituation}
The last few years have seen significant advances in non-speech audio processing, as popular deep learning architectures, developed in the speech and image processing communities, have been ported to this relatively understudied domain. However, these data-hungry neural networks are not always matched to the available training data in the audio domain. While unlabeled audio is easy to collect, manually labelling data for each new sound application remains notoriously costly and time-consuming. Therefore a considerable amount of research in recent years focuses on the application of unsupervised machine learning in the audio domain.
\newline
\newline
When listening to music, distinguishing between music genres is for humans, from easy to difficult, depending on how similar the given genres are. It is rather simple to distinguish music songs, for example, of classical music and techno or country and hip-hop rap. However, the difficulty increases when songs within the same genre will need to be compared. When it comes to finding the similarity between songs or to sort a list of songs by their similarity, even humans fail to succeed. 
\newline
\newline
This thesis seeks to alleviate this problem by developing an alternative learning strategy that exploits basic semantic properties of sound that are not grounded by an explicit label. This alternative will then be applied to a noise detection dataset and a music dataset to show the applicability and performance of the approach. Further, it attempts to show that the representation learned, provides a similarity space for both datasets, which can then be used to compare specific audios, sort them by similarity or find similarities between categories.

\section{Task definition}
\label{sub:Task-Definition}
The goal of this thesis is to adapt Tile2Vec, an image embedding method, to audio streams and evaluate its performance on the noise detection dataset \flqq SINS, DCASE 2018: task 5\frqq. 
\newline
\newline
To achieve this goal, a model, which transforms the audio signals into embeddings, needs to be trained first. Then to evaluate the performance on the DCASE dataset, a simple classifier will be used, which receives an embedding and classifies it into one of the given classes within the dataset. The light classifier will be trained, and the accuracy of the resulting architecture will be compared to the published results of the DCASE 2018 Task 5 challenge.
\newline
\newline
This method is to be applied exploratively to different kinds of music, whereby a data set of 18h DJ music is available. The resulting clusters will be investigated qualitatively, by using clustering techniques as well as manual evaluation.

\section{Target setting}
\label{sub:Target-Setting}
As stated in \fullref{sub:Task-Definition}, the goal of the thesis is to adapt Tile2Vec to audio streams and evaluate its performance. The thesis is categorised as an exploratory innovation/research project since its main focus is to evaluate if the task can be completed and how well it performs. Another target is to further give an overview of how advanced the current research is in the section of unsupervised machine learning with audio and as well give an outlook on further projects, to explore and evaluate different approaches.

\section{Preliminaries}
\label{sub:Preliminaries}
The focus of the thesis is exclusively on the architecture of the embedding model and the preprocessing of the audio signal. Clustering and classification are explicitly excluded from the research. Instead, proven and established standard approaches are to be used.
\newline
\newline
Clustering will only be used for evaluating the resulting clusters within the embedding space of the music dataset. For the classification, a simple two-layered dense model is used and should only to give some insights on how well a simple model performs on the resulting embedding space. The idea is to show that there is a specific performance gain when using the embedding space.
\newline
\newline
Optimization methods for training neural networks are explicitly excluded from the research. Instead, state-of-the-art techniques implemented in Tensorflow are used.


\chapter{Related Work}
\label{ch:Related-Work}

This chapter introduces and explains terms and concepts which will be used throughout the thesis. Thus, this section can be used as a reference guide to fill in any gaps and to understand the relationships between the individual concepts better.

\section{Technological fundamentals}
\label{sec:Technological-Fundamentatls}

This section deals with the essential technological fundamentals of the project. These basics contain mainly definitions of terms, which will be used repeatedly in the following chapters. Most of these principles are superordinate terms, which are required to understand the detailed concepts in \fullref{sec:Technological-Fundamentatls}.

\subsection{Tensorflow}
\label{sub:Tensorflow}

\section{Technical concepts}
\label{sec:Technical-Concepts}

In this section, technical concepts are explained in more detail, which was used in the thesis. These concepts are mostly very complex and are, therefore only touched upon so that the further conclusions of this work can be understood comprehensibly.

\subsection{Word2Vec}
\label{sub:word2wec}

\subsection{Triplet Loss}
\label{sub:Triplet-Loss}

\subsection{Mel Spectogram}
\label{sub:Mel-Spectogram}

\subsection{Fourier Transformation}
\label{sub:Fourier-Transformation}

\subsubsection{Fast Fourier Transformation}
\label{subsub:Fast-Fourier-Transformation}

\subsubsection{Short Time Fourier Transformation}
\label{subsub:Short-Time-Fourier-Transformation}

\subsection{$\mu$-law Transformation}
\label{sub:Mu-Law-Transformation}

\section{Status in relation to project}
\label{sec:Status-Relation-Project}


\chapter{Ideas and concepts}
\label{ch:Ideas-Concepts}
This chapter explains the different ideas and concepts used to realise the thesis. Some ideas were discovered during the research and were already introduced in the sections of related work (\ref{ch:Related-Work}) or were introduced in other approaches (\ref{sec:Other-Approaches}). Others were developed especially for this project and their application, as well as their advantages, will be proven in this work.
\newline
\newline
It is important to know that in the thesis, two distinct datasets are used to evaluate the application of the developed architecture in two domains. Due to that, many of the sections in this chapter are divided into two groups, one for the \fullref{sub:DCASE-Task-Dataset} and one for the music dataset (\ref{sub:Music-Dataset}).

\section{Data preprocessing}
\label{sec:Data-Preprocessing}
Data preprocessing is the process of processing the entire dataset before it is fed into the input pipeline (\ref{sec:Input-Pipeline}), either manually or automatically. Both datasets used in the project consist of audio files, which are grouped together by their corresponding label. Thus the structure of both datasets is very similar. For this thesis, it was decided that data preprocessing was not necessary for the \fullref{sub:DCASE-Task-Dataset}. However, for the music dataset (\ref{sub:Music-Dataset}), a minimal manual preprocessing is needed to improve the performance of the pipeline.
\newline
\newline
The \fullref{sub:DCASE-Task-Dataset} was already preprocessed by the organisers of the challenge. Which means that there are no audio files which contain multiple labels, are corrupted or invalid. The organisers of the challenge further split the sessions of recordings into 10s samples, which provides an easy and fast to use dataset. Therefore the preprocessing is neglected.
\newline
\newline
The music dataset (\ref{sub:Music-Dataset}) consists of different songs grouped by their respective genre. The audio samples in this dataset are given in the format of \texttt{.mp3} files, which is not that fast to process as for example \texttt{.wav} formatted audios. Therefore all of the \texttt{.mp3} audios were converted using \texttt{ffmpeg} to \texttt{.wav} files, with the simple script \ref{code:Convert-mp3-wav}.
\begin{code}[htbp]
\begin{minted}{bash}
for i in *.mp3; 
do ffmpeg -i "$i" "${i%.*}.wav";
done;
\end{minted}
\caption{Convert \texttt{.mp3} files to \texttt{.wav} files using \texttt{ffmpeg}}
\label{code:Convert-mp3-wav}
\end{code}
Further, the audios of the music dataset will be trimmed within the input pipeline (\ref{sec:Input-Pipeline}), because they should not contain any silence at the start or end of a song since this could lead to misclassified segments and clusters of silence, which should be omitted.

\section{Feature extraction}
\label{sec:Feature-Extraction}
Feature extraction, sometimes called feature representation, is the process of extracting the relevant features from the raw data; in this case, audio files. The extraction process is done within the input pipeline (\ref{sec:Input-Pipeline}). For both datasets, the same feature representations are used. Three different extractions are evaluated and are used as a hyperparameter which can be tuned to achieve the best possible performance. The optimal feature representation for the two datasets may be different since their characteristics are quite different. However, research has to show if that is the case.
\newline
\newline
The first approach is to use the audio files as they are, more precisely to use the raw waveform as input to the model. This is the most straightforward approach and as well the most lightweight to compute. It will be used to evaluate how good a model can be trained without any further feature extraction. The approach to use the raw waveform would be the simplest one to implement in a real-world application, because there is no complex computation needed, and can therefore as well be used on small devices, such as mobile phones or Raspberry Pis. However, when using the raw waveform of an audio segment, the preprocessing task of the input pipeline will get omitted, but the model gets a lot bigger since the input size is much larger than when using a feature representation. For example, when using the raw waveform of a 10s segment with the sample rate of 16'000, the input size gets 160'000, which is very large in comparison to the size when using a feature representation.
\newline
\newline
The second approach is to calculate the log-Mel spectrogram, which is outlined in subsection Mel spectrogram (\ref{sub:Mel-Spectrogram}), where the last step of calculating the \gls{DCT} is omitted. The process described was introduced to mimic the human hearing, but the last step, calculating the \gls{DCT}, was proposed because many machine learning models struggled with the highly correlated data, such as the log-Mel spectrogram, and thus a linear transformation (\gls{DCT}) was added to decorrelate the feature. Contemporary machine learning models such as \glspl{CNN} or \glspl{GRU} do not struggle with correlated data and perform even better when they can decorrelate the features themselves. Therefore this approach should give useful insight, weather the decorrelation can be trained and is still needed in present models. This approach is widely used in the audio domain because it is less computing extensive than calculating the \glspl{MFCC} and it can achieve almost the same, or an even better, accuracy as a model trained with the \glspl{MFCC} as their feature representation because it uses the same computations and additionally preserves more information.
\newline
\newline
The third and final approach is to calculate and use the \glspl{MFCC}, which is described in subsection Mel spectrogram (\ref{sub:Mel-Spectrogram}). This feature extraction method is used mainly in extensive audio applications, such as automatic speech and speaker recognition, where a lot of computing power is available. These features are mostly used because of their nature, that they represent sound like the human auditory system does, in a compact matter.
\newline
\newline
All of the features stated above are two-dimensional data, and therefore, standard image processing architectures can be adapted to audio processing. This adaption of image processing architectures to audio has successfully be shown in recent years of research \footcite{dai_acoustic_2016} \footcite{takahashi_deep_2016} \footcite{purwins_deep_2019} \footcite{lee_samplecnn_2018} \footcite{pons_timbre_2017}. 

\section{Triplet selection}
\label{sec:Triplet-Selection}
The triplet selection describes the process of selecting a neighbouring and opposite sample from a given anchor sample. The idea of triplet selection and the motivation behind it is described within section triplet loss (\ref{eq:Triplet-Loss}). The procedure is split into two parts, the selection of the neighbouring and opposite tile.
\newline
\newline
The straightforward way to triplet selection, as described in \ref{eq:Triplet-Loss} is to select the anchor and the neighbour to be part of the same label and to select the neighbour to have a different label than the other. However, this approach can not be used in an unsupervised learning technique because the data does not have any information about the underlying label. Due to that, a different, unsupervised approach to triplet selection has to be chosen.
\newline
\newline
For the triplet selection of the \fullref{sub:DCASE-Task-Dataset}, all of the 10s audio files have to be split into a specified \textit{sample length} (e.g. 1s, 2s) first. This results in different segments belonging to the same audio file, which are referred to as segments and denoted as $x$. The sample length specifies the length of all the segments. Therefore a triplet consists of three different segments $(x_a, x_n, x_o)$.
\newline
For the anchor tile $x_a$, a random segment is chosen. For the neighbouring tile $x_n$, a segment will be selected which belongs to the same audio file as the anchor, where the selection has to be done within a predefined range, \textit{neighbouring selection range} (e.g. 2s, 4s), of the anchor tile $x_a$. It is important to note, that it does not make any difference if the tile is being sampled from before or after the anchor segment. The segments $x_a$ and $x_n$ have to be distinct from each other and chosen within the specified \textit{neighbouring selection range}.
\newline
The opposite tile $x_o$ will be chosen at random from a different audio sample, where the selection of the segment is as well performed at random. The only constraint is that the opposite tile can not be chosen from the same audio file as the anchor and the neighbour.
\newline
\newline
For the music dataset (\ref{sub:Music-Dataset}), the triplet selection is made relatively similar to the one described in the paragraph above for the \fullref{sub:DCASE-Task-Dataset}. The only difference between the two methods is that for the music dataset, the audio files are of variable length. However, this does not make any difference in the selection process. The only difference between the two methods is that maybe the \textit{sample length} and \textit{neighbouring selection} range will need to be increased due to the increased audio file length, but this has to be shown in the experiments.

\section{Input pipeline}
\label{sec:Input-Pipeline}
The input pipeline is responsible for creating the dataset, which will then be fed into the model by batches. For this project, the input pipeline is implemented using the TensorFlow \texttt{tf.data} API\footnotemark, along with a generator, which loops over the whole dataset and creates entries on the fly. The reason why a generator function is used, is that the entire dataset can not be loaded into memory, due to the large size of it. The entire process of the input pipeline is illustrated in figure \ref{fig:Input-Pipeline-Visualisation}.
\footnotetext{\url{https://www.tensorflow.org/api_docs/python/tf/data}}
\newline
\newline
The feature extraction (\ref{sec:Feature-Extraction}) and the triplet selection (\ref{sec:Triplet-Selection}) of the audio files is done on the fly within the input pipeline (\ref{sec:Input-Pipeline}). The advantage of this method is that the pipeline can be adapted fairly quick for other datasets since the only input the pipeline requires is the raw waveform of an audio signal of any kind. 
\newline
\newline
The first step of the input pipeline is to loop over the corresponding generator function of the specified dataset, which loops over the entire dataset and returns the current sample of the dataset. For each one of the returned samples, a triplet, consisting of an anchor, neighbour and an opposite audio segment, is created by using the datasets method to perform the selection. The triplet, which will be returned from the dataset consists of indices of the audio and segment for each one of the three segments. Then the input pipeline uses these indices to extract the corresponding audio segment out of the dataset and yields them back to the \texttt{tf.data.Dataset}. It is important to note that all the operations listed after this paragraph are done on the whole batch.
\newline
\newline
After the dataset gets an entire batch of audio segments, the feature extraction will be performed. The feature extraction is implemented using the vectorised \texttt{.map} function of the class \texttt{tf.data.Dataset}\footnotemark. The \texttt{.map} function applies a given function to each element within the dataset, which in the case of this thesis will be used to extract the features from the audio. Hence a \flqq feature extractor\frqq \ has to be provided to the input pipeline, which will extract a certain feature from the audio as stated in \ref{sec:Feature-Extraction}. If no extractor is passed to the pipeline, the raw waveform of the signal is used. 
\footnotetext{\url{https://www.tensorflow.org/api_docs/python/tf/data/Dataset}}
\newline
\newline
After the extraction, the dataset is shuffled and returned to the training process. In the training loop, the dataset can now be iterated over, which yields back a batch of triplets, consisting of the extracted audio segment. This batch can then be used to train the model. The dataset is therefore similar to a Python iterator.
\newline
\newline
The advantages of this method are that this creates the dataset dynamically, because the dataset only contains the current batch along with some prefetched entries, which are entries from the next batch so that the \gls{GPU} can be fully utilised.
\begin{figure}[htbp]
	\centering
	\includegraphics[scale=0.4]{img/Input_Pipeline_Visualisation.png}
	\caption{Visualisation of the input pipeline process}
	\label{fig:Input-Pipeline-Visualisation}
\end{figure}

\section{Models}
\label{sec:Models}
This section describes the overall architecture of the models used in the thesis. The idea of the thesis is to evaluate different kinds of architectures and compare their results to find the optimal model for the thesis.
\newline
\newline
The first architecture, which will be evaluated in the thesis are \glspl{CNN} (\ref{sub:Convolutional-Neural-Network}). They are used in a lot of audio applications because of their simplicity and robustness. A considerable amount of research in the past couple of years has shown the overall success of these models. Therefore it makes sense to evaluate these type of models extensively. 
\newline
\newline
Another widely used model architecture is the \gls{RNN}, where \glspl{GRU} (\ref{sub:Gated-Recurrent-Unit}) is one of the most prominent implementations. Their application primarily lies in the text-domain. However, in recent years they have shown their success within the audio domain, mainly because of the sequential nature of the audio signal.
\newline
\newline
The most prominent approach in audio research is the combination of \glspl{CNN} and \glspl{RNN}. The architecture consists of multiple convolution layers, and a single or multiple recurrent layers. The idea of these models is that they first reduce the input data to a specific lower-dimensional representation and then use recurrent models to make use of the sequential nature of the audio. Therefore the best ideas of both architectures are used within these \glspl{CRNN}.

\section{Application to music}
\label{sec:Application-Music}
All of the concepts mentioned above are designed to work with both datasets. Thus the application to music is described in each of the sections above separately. But to summarise, to adapt the model from the noise detection dataset to the music dataset, the same architecture will be used. This includes using the same triplet selection process. However, almost inevitably, the segments will need to be increased. Then to use the same input pipeline, with the same feature representation. This results in almost the same architecture as for the \gls{DCASE} dataset, with a few exceptions.

\section{Metrics}
\label{sec:Metrics}
Meaningful metrics need to be used to evaluate and compare the performance of machine learning models. In supervised machine learning, most of the metrics focus on how well the model predicts the actual value from the input, but in the unsupervised setting, finding meaningful metrics is a lot harder. Hence the metrics for the embedding model and the classifier are quite different because the embedding architecture is unsupervised while the classifier is supervised. Therefore this section is split into two parts, metrics for the embedding space (\ref{sub:Metrics-Embedding-Space}) and metrics for the classifier (\ref{sub:Metrics-Classifier}), which both aim to give valuable insights into the performance. All of the mentioned metrics below are monitored using the Tensorboard visualisation toolkit\footnote{\url{https://www.tensorflow.org/tensorboard}}.

\subsection{Embedding space}
\label{sub:Metrics-Embedding-Space}
As mentioned before, it is not easy to find meaningful metrics for the embedding space because of its unsupervised nature. However, since the thesis focuses on the application of triplet loss in the unsupervised setting one of the most straight forward metrics to use is to monitor the triplet loss value, which gives insight if the model is training and learning a representation over time. Thus this value should be minimized over time. The triplet loss value, which will be monitored, results out of the equation \ref{eq:Triplet-Loss}. Since the idea of triplet loss is to minimize the distance between the anchor and the positive sample and to maximize the distance between the anchor and the negative sample, it makes sense to monitor them. Both values are already computed when computing the triplet loss and only have to be monitored. As a distance measure between the embeddings, the squared euclidean distance, given by equation \ref{eq:Euclidean-Distance}, is used. Monitoring these distances, give valuable insights if the model is capable of satisfying the triplet loss constraint or not. All of these three metrics are monitored both on the training and evaluation dataset.s
\newline
\newline
The embedding space can further be thought of as a clustering task, where the goal is to cluster similar samples in the near-by region. Since the evaluation set contains the ground truth of each sample, most clustering metrics can be used to monitor the progress of the embedding space. 
\newline
\newline
One of the most straightforward metrics to use for evaluation and comparison of the resulting embedding spaces is the distances between the clusters. They can be computed by calculating the distance from each centroid of a cluster to every other. The metric gives an insight into how distant the clusters of each label are. It can be displayed in two different ways, as different graphs and as a distance matrix in the form of an image. Such a representation in the form of an image is shown in figure \ref{fig:Distance-Matrix}. However, it is important to say that this metric is very vulnerable for outliers, which means that the centroid can vary very broadly if one of the clusters contains many outliers.
\newline
\newline
Further, a popular clustering metric called the silhouette coefficient, is used, where a higher score relates to a model with better-defined clusters. The score is bounded between $-1$ for incorrect clustering and $+1$ for highly dense clustering. Scores around zero indicate overlapping clusters. The score is higher when clusters are dense and well separated, which relates to a standard concept of a cluster. The silhouette coefficient is defined for each sample and is composed of two scores:
\begin{itemize}
\setlength\itemsep{0em}
    \item $a$: The mean distance between a sample and all other points in the same class
    \item $b$: The mean distance between a sample and all other points in the next nearest cluster
\end{itemize}
The silhouette coefficient $s$ for a single sample is then given by equation \ref{eq:Silhouette-Coefficient}. For a set of samples, the silhouette coefficient is given as the mean of the silhouette coefficient for each sample.
\myequations{Silhouette coefficient for a single sample}
\begin{equation}
    \centering
    \begin{gathered}
        s = \frac{b - a}{max(a, b)}
    \end{gathered}
    \label{eq:Silhouette-Coefficient}
\end{equation}
Both of the metrics above are only computed on the evaluation dataset, since monitoring it on the training set would not provide more insight.
\begin{figure}[htbp]
	\centering
	\includegraphics[scale=0.25]{img/Distance_Metric.png}
	\caption{Distance matrix from each centroid to every other}
	\label{fig:Distance-Matrix}
\end{figure}

\subsection{Classifier}
\label{sub:Metrics-Classifier}
For the classifier, standard supervised metrics are monitored, such as loss and accuracy. The loss function used to train the model is the sparse categorical cross-entropy loss. Therefore this loss function will be monitored to show if the model minimises this function over time. As an accuracy metric, the sparse categorical accuracy is monitored to show how well the classifier predicts the inputs.
\newline
\newline
The usage of the accuracy does not work well with highly unbalanced datasets such as the \fullref{sub:DCASE-Task-Dataset}. Therefore another metric needs to be used, which works well with this kind of dataset. The organisers of the \gls{DCASE} task 5 challenge used the macro-averaged F1 score to evaluate the models' performance. This score satisfies the constraint of working well with unbalanced data and is therefore used to monitor the accuracy of the classifier as well. Furthermore, it is used to compare the results accomplished in the thesis to the results by other models submitted in the challenge. The F1 score is given by equation \ref{eq:F1-Score} and to get the macro-averaged F1 score, the metric has to be calculated for each label, and then the unweighted mean is taken.
\myequations{F1 score}
\begin{equation}
    \centering
    \begin{gathered}
        \text{F1} = 2 \cdot \frac{\text{precision} \cdot \text{recall}}{\text{precision} + \text{recall}} \\
        \text{precision} = \frac{\text{TP}}{\text{TP} + \text{FP}} \\
        \text{recall} = \frac{\text{TP}}{\text{TP} + \text{FN}} \\
    \end{gathered}
    \label{eq:F1-Score}
\end{equation}
where:
\begin{conditions*}
    \text{TP} & true positive \\   
    \text{TN} & true negative \\ 
    \text{FP} & false positive \\ 
    \text{FN} & false negative \\ 
\end{conditions*}
\noindent
All of the metrics for the classifier are both monitored for the training as well as the evaluation dataset.

\chapter{Method}
\label{ch:Method}
In this chapter, the methods, which are used in the thesis, will be described. This includes the project plan (\ref{sec:Project-Plan}), where the whole project plan is explained in detail, the procedure model (\ref{sec:Procedure-Model}), which is used to realise the thesis and the risk analysis (\ref{sec:Risk-Analysis}). This chapter further describes how the success of the project is measured and compared (\ref{sec:Evaluation}) as well as how to test the individual components (\ref{sec:Testing}). Finally, it contains information about the overall project structure (\ref{sec:Project-Structure}).

\section{Project plan}
\label{sec:Project-Plan}
Within this section of the thesis, the project plan is shown, based on figure \ref{fig:Project-Plan}. The project is divided into four different phases. At the end of each phase, there is a milestone, which will review the process within the phase and provide an outlook on the next phase. The first phase is doing research, where many resources are gathered, and essential pieces of information are extracted and documented. The goal of this stage is to get a better understanding of the concepts, which already exist. It further aims to get insights into the current status of the research in the audio domain. Afterwards, the realisation of the project takes place, where the models, input pipeline and Tile2Vec will be implemented and tested. Then different experiments will be conducted, to validate the realisation and experiment with different hyperparameters. At last, the documentation will be finalised and proofread. There is also a one week buffer before the finalisation of the documentation, which will be used for unpredictable problems. The chapters of the documentation will be continuously written during each phase.
\newline
\newline
The figure \ref{fig:Project-Plan} illustrates the project plan, where the phases are shown in orange, tasks within each phase are blue, and milestones are red diamonds. Each dotted column represents one week within a specific month (e.g. Feb, Mar). A task further has information about how far it is into completion.

% set page layout to portrait and save layout for later
\storeareas\defaultpagestyle
\KOMAoptions{pagesize,paper=landscape,DIV=20}
\storeareas\landscapevalues

\begin{figure}
\centering
     \begin{ganttchart}[%Specs
     x unit = 0.9cm,  %<---------------------- New x unit 
     y unit title=0.4cm,
     y unit chart=0.5cm,
     vgrid, hgrid,
     title height=1,
     title label font=\bfseries\footnotesize,
     progress=today,
     today=14,
     today label=Current Week,
     bar/.append style={fill=blue!50},
     group/.append style={draw=black, fill=orange!40!white},
     milestone/.append style={draw=black, fill=red!40!white, xscale=0.55}, % 0.5/0.9 ≈ 0.5555
     milestone progress label node/.append style={right=0.2},
     bar incomplete/.append style={fill=none},
     group incomplete/.append style={draw=black,fill=none},
     bar height=0.7,
     group right shift=0,
     group top shift=0.5,
     group height=.3,
     group peaks width={0.2},
     inline]{1}{16}
    %labels

    %\gantttitle{bachelor thesis - deep embedded music}{18}\\
    \gantttitle[]{2020}{16} \\                
    \gantttitle{Feb}{2}
    \gantttitle{Mar}{4}
    \gantttitle{Apr}{4}
    \gantttitle{May}{4}
    \gantttitle{Jun}{2}\\
    
    % 1. Phase: Research
    \ganttgroup[inline=false]{Research}{1}{4}\\ 
    \ganttbar[progress=100,inline=false]{Dataset}{1}{1}\\
    \ganttbar[progress=100,inline=false]{Audio processing}{1}{1}\\
    \ganttbar[progress=100,inline=false]{Triplet loss}{2}{2}\\
    \ganttbar[progress=100,inline=false]{Tile2Vec}{3}{3}\\
    \ganttbar[progress=100,inline=false]{Evaluate research}{4}{4}\\
    \ganttmilestone[inline=false]{Research finished}{4} \\ % M1
    
    % 2. Phase: Realization
    \ganttgroup[inline=false]{Realisation}{5}{8} \\
    \ganttbar[progress=100,inline=false]{Project setup}{5}{5} \\
    \ganttbar[progress=100,inline=false]{Input pipeline}{5}{5} \\
    \ganttbar[progress=100,inline=false]{Default model architectures}{6}{6} \\
    \ganttbar[progress=100,inline=false]{Evaluation workflow}{6}{6} \\
    \ganttbar[progress=100,inline=false]{Unit tests}{6}{6} \\
    \ganttmilestone[inline=false]{Project Setup finished}{6} \\ % M2
    \ganttbar[progress=100,inline=false]{Tile2Vec implementation}{7}{7} \\
    \ganttbar[progress=100,inline=false]{Architecture for experiments}{8}{8} \\
    \ganttmilestone[inline=false]{Realisation finished}{8} \\ % M3
    
    \ganttmilestone[inline=false]{Interim presentation}{9} \\ % M3
    
    % 3. Phase: Experiments
    \ganttgroup[inline=false]{Experiments}{9}{12} \\
    \ganttbar[progress=100,inline=false]{Conduct experiments}{9}{9} \\
    \ganttbar[progress=100,inline=false]{Validate embeddings}{10}{11} \\
    \ganttmilestone[inline=false]{Experiments finished}{11} \\ % M4
    \ganttbar[progress=100,inline=false]{Visualise embeddings}{12}{12} \\
    
    % Buffer
    \ganttgroup[inline=false]{Buffer}{13}{13} \\
    
    % 4. Phase: Documentation
    \ganttgroup[inline=false]{Finalise documentation}{14}{16} \\
    \ganttbar[progress=50,inline=false]{Create pitch video and web abstract}{14}{14} \\
    \ganttbar[progress=100,inline=false]{Finalise documentation}{14}{15} \\
    \ganttbar[progress=20,inline=false]{Proofread documentation}{16}{16} \\
    \ganttmilestone[inline=false]{Thesis submission}{16}
\end{ganttchart}
\caption{Project plan}
\label{fig:Project-Plan}
\end{figure}

% set page back to portrait
\clearpage
\defaultpagestyle

\section{Procedure model}
\label{sec:Procedure-Model}
The waterfall model is a breakdown of project activities into linear sequential phases, where each phase depends on the deliverables of the previous one and corresponds to a specialisation of tasks. For this project, a custom waterfall model is used, which differs quite profoundly from Royce's original waterfall model. The procedure model is shown in figure \ref{fig:Project-Plan}.
\newline
\newline
The main reason why a linear procedure model is used and not an iterative such as SCRUM is, that, the project is an innovation/research project, where the main focus is on conducting experiments and searching for new findings for the problem definition. Therefore, it is obvious which tasks need to be completed during each phase and what each one of them has to deliver before going to the next phase. Another reason why the waterfall model is chosen is that the project is done only by a single person and therefore there is no waiting for others to complete their task before going to the next phase since all of the work carried out is done by one person. Another benefit compared to an iterative procedure model is, that the project is already clearly defined by the start of the project and the artefacts, which will need to be delivered by the end of the project, is as well given at the start. Therefore the status of the project is evident at any point during the project.

\subsection{Project phases}
\label{sec:Project-Phases}
\begin{table}[htb]
    \centering
    \caption{Phases in the project plan}
	\label{tab:Phases}
    \begin{tabular}{p{.05\textwidth} | p{.15\textwidth} | p{.70\textwidth}}
        \toprule
        \textbf{\#} & \textbf{Phase} & \textbf{Description} \\ 
        \midrule[1pt]
        P1 & Research & The first phase of the project is to do research on the field and to describe it in chapter \ref{ch:Related-Work}. The deliverable of this phase is the finished draft of the chapter related work (D1). \\
        \hline
        P2 & Realisation & The second phase is relatively large and is used to implement the codebase of the project, which includes the code for training, testing and evaluating. The deliverable of this phase is a fully functional and tested codebase to train and evaluate embedding models (D2 \& D3).\\
        \hline
        P3 & Experiments & In the third phase, all of the experiments will be conducted and evaluated. Within this phase, there is the possibility that the codebase will be adjusted to optimise the performance of the experiments. The deliverables of this phase are two fully trained models, one for the noise detection dataset and one for the music dataset (D4).\\
        \hline
        P4 & Buffer & The buffer phase is used to provide an extra one week buffer if there is a  delay with one of the phases before. Therefore, this phase is variable and does not have clear deliverables at the start of the project. If the buffer is not needed, it will be used to extend the last phase P5. \\
        \hline
        P5 & Finalise documentation & The last phase is used to finalise the thesis, create the web abstract and the pitching video. These are all artefacts, which are needed to submit the thesis. The deliverables are, therefore, the finished documentation, the pitching video and the web abstract (D5). \\
        \bottomrule
    \end{tabular}
\end{table}
\noindent
The project is grouped into four phases, \textit{research}, \textit{realisation}, \textit{experiments} and \textit{finalise documentation}. However, there is as well an additional \textit{buffer} phase which will be used where ever there is a need for an additional week to finish the phase. All of these phases consist of multiple tasks which need to be completed before going to the next step. Each phase is only then completed when the deliverables for the phase are delivered and reviewed. The table \ref{tab:Phases}, describes each phase in more detail and further describes the deliverables of each one of them.

\subsection{Deliverables}
\label{sec:Deliverables-Project}
\begin{table}[htb]
    \centering
    \caption{Artefacts to deliver}
	\label{tab:Deliverables}
    \begin{tabular}{p{.05\textwidth} | p{.25\textwidth} | p{.60\textwidth}}
        \toprule
        \textbf{\#} & \textbf{Deliverable} & \textbf{Description} \\ 
        \midrule[1pt]
        D1 & Research \gls{DSP}, \gls{NN} and state of the art (related work) & These artefacts need to be delivered in the form of chapter \ref{ch:Related-Work} (related work), which needs to be reviewed and accepted by the advisor. \\
        \hline
        D2 & Evaluation workflow &  Implementation of the evaluation workflow to validate the model performance on either of the datasets. \\
        \hline
        D3 & Implementation codebase & Implementation of the codebase for training the model. This consists of the realisation of the input and training pipeline. \\
        \hline
        D4 & Experiments (study docs) &  For each of the conducted experiments, a separate study doc needs to be delivered, which will be used to provide detailed information about the experiments and its outcome. \\
        \hline
        D5 & Documentation & The documentation needs to be delivered at the end of the project along with a web abstract and a pitching video, which will be needed to complete the upload of the thesis. \\
        \bottomrule
    \end{tabular}
\end{table}
\noindent
Deliverables are artefacts which need to be delivered at the end of each phase to complete it. They will be examined and reviewed before going to the next phase. All of the deliverables in the project are shown in table \ref{tab:Deliverables}. Each deliverable corresponds to a particular phase, and their affiliation is shown in table \ref{tab:Phases}.

\subsection{Milestones}
\label{sec:Milestones-Project}
\begin{table}[htb]
    \centering
    \caption{Milestones overview along with their corresponding date}
	\label{tab:Milestones}
    \begin{tabular}{p{.05\textwidth} | p{.45\textwidth} | p{.20\textwidth} | p{.15\textwidth}}
        \toprule
        \textbf{\#} & \textbf{Milestone} & \textbf{Deliverables} & \textbf{Date} \\ 
        \midrule[1pt]
        M0 & Project start & - & 17.02.2020\\
        \hline
        M1 & Research finished & D1 & 15.03.2020\\
        \hline
        M2 & Project setup finished & D2 & 29.03.2020\\
        \hline
        M3 & Realisation finished & D3 & 12.04.2020\\
        \hline
        M4 & Interim presentation & - & 21.04.2020\\
        \hline
        M5 & Experiments finished & D4 & 03.05.2020\\
        \hline
        M6 & Thesis submission & D5 & 05.06.2020\\
        \bottomrule
    \end{tabular}
\end{table}
\noindent
There is a specific milestone at the end of each phase \ref{tab:Phases}, which reviews the deliverables and the work done within the phase. The milestone review further aims to provide an outlook on the next phase and the current status of the project. The review is done by writing a report for each milestone, which can be found in the appendix milestone reports (\ref{app:Milestone-Reports}). Each of these reports gives insight about the status of the project by answering the questions \textit{what was done since the last reporting?}, \textit{what is the state of progress of the project} and \textit{what are the top three risks?}, which further includes the planned measures to take. The reports provide valuable insights into the projects current status and are therefore used to plan the next phases. Table \ref{tab:Milestones} shows all the milestones with the corresponding data assigned to it. The milestones are illustrated in the full project plan \ref{fig:Project-Plan} in red.
\newline
\newline
It is important to note that the milestone \textit{M4: Interim presentation} does not conclude a phase. It is purely used to show wherein the project the interim presentation is held and what needs to be delivered until then. Therefore, for this milestone, no milestone report is being written.

% set page layout to landscape
\clearpage
\landscapevalues

\section{Risk analysis}
\label{sec:Risk-Analysis}
\begin{table}[htb]
    \centering
    \caption{Risk analysis}
	\label{tab:Risk-Analysis}
    \begin{tabular}{p{.05\textwidth} | p{.30\textwidth} | p{.30\textwidth} | p{.30\textwidth}}
        \toprule
        \textbf{\#} & \textbf{Risk} & \textbf{Consequences} & \textbf{Planned measurements} \\ 
        \midrule[1pt]
        R1 & The codebase has flaws in it. & The experiments fail, the model does not train, and the codebase needs to be changed during a late stage. & Using a test-first approach with automated unit tests. \\
        \hline
        R2 & The model fails to find an underlying structure of the data. & The project fails in a late-stage, and all of the conducted experiments showed that the model failed. & Implementing small models and train them in an early stage to show the possibility of solving the problem.  \\
        \hline
        R3 & The training of the model fails because there is not enough computing power. & The large models needed to find an optimal embedding space can not be trained. & Early testing on the available computing resources and if more needed, rent \gls{GPU} instances on \texttt{vast.ai}. \\
        \hline
        R4 & Optimal hyperparameters for the model can not be found. & An optimal embedding space can not be found, and all of the experiments result in suboptimal results. & The first results and hyperparameters are discussed with the advisor, and thereafter the advisor is contacted if help is needed. \\
        \hline
        R5 & The model fails to find an optimal embedding space for the music dataset & The second part of the project fails, and the unsupervised triplet loss approach will be discarded. & This is a limitation of the project and is part of the research process. \\
        \hline
        R6 & The project can not be completed due to time limitations. & Some of the deliverables can not be delivered at the end of the project. The goal of the project is not reached. & Early start of the documentation, consequent documenting, meetings with the advisor and rapid changes if a loss of time is observed. \\ 
        \bottomrule
    \end{tabular}
\end{table}
\noindent
This section shows the risk analysis done to show the possible risks when conducting the project. Table \ref{tab:Risk-Analysis} shows the risk analysis, which consists of the risk, the possible consequences and the planned measurements. The table \ref{tab:Risk-Matrix} shows the residual risk matrix, which provides an overview of the possibility of the occurrence of each risk. For all of the risks, the measurements are planned to get them as low as reasonably possible. Therefore the goal is to have all of the risks in a green section, which indicates that it is acceptable. Yellow represents acceptable. However, further risk reduction should be made. Furthermore, red indicates not acceptable, and a risk reduction is profoundly needed. The risk matrix \ref{tab:Risk-Matrix} show the probability of the occurrence of each risk when taking the counter measurements already into account.

% set page back to portrait
\clearpage
\defaultpagestyle

\begin{table}[htb]
\centering
\scriptsize
\caption{Residual risk matrix}
\label{tab:Risk-Matrix}
\begin{tabular}{|p{3cm}|p{2cm}|p{1.6cm}| p{1.8cm} |p{1.8cm}| p{1.6cm}|}
    \hline \diagbox[innerwidth=3cm]{Frequency}{Consequence} & 1-Very Unlikely & 2-Remote & 3-Occasional & 4-Probable & 5-Frequent\\ [10pt]
    \hline 4-Catastrophic & \cellcolor{yellow!40!white} & \cellcolor{red!40!white} & \cellcolor{red!40!white} & \cellcolor{red!40!white} &\cellcolor{red!40!white} \\ [10pt]
    \hline 3-Critical & R2 \cellcolor{green!40!white} & R6 \cellcolor{yellow!40!white} & R5\cellcolor{yellow!40!white} & \cellcolor{red!40!white} &\cellcolor{red!40!white} \\ [10pt]
    \hline 2-Major & \cellcolor{green!40!white} & R3 \cellcolor{green!40!white} & R4 \cellcolor{yellow!40!white} & R1 \cellcolor{yellow!40!white} &\cellcolor{red!40!white} \\ [10pt]
    \hline 1-Minor & \cellcolor{green!40!white} & \cellcolor{green!40!white} & \cellcolor{green!40!white} &\cellcolor{yellow!40!white} &\cellcolor{yellow!40!white} \\ [10pt]
    \hline
\end{tabular}
\end{table}

\section{Evaluation}
\label{sec:Evaluation}
The evaluation is done for both used datasets separately, and in a relatively different way, therefore both evaluations are described in different subsections, the evaluation of the DCASE dataset (\ref{sub:Eval-DCASE}) and the evaluation of the music dataset (\ref{sub:Eval-Music}).

\subsection{DCASE 2018 - Task 5 dataset}
\label{sub:Eval-DCASE}
\begin{figure}[ht]
    \centering
    \begin{tikzpicture}[standard/.style={inner sep=0pt,align=center,draw,text height=1.25em,text depth=0.5em},
    decoration={brace}]
    \node[text width=8cm, yshift=1cm, standard] (Trd)  {development dataset};
    \node[right=0.5em of Trd, standard, text width=4cm] (Ted)  {evaluation dataset};
    \node[fit=(Trd)(Ted), yshift=1cm, standard] (Ald)  {all data from the DCASE challenge}; 
    \node[anchor=north west, standard, text width=5.8cm, fill=yellow!30!white, yshift=-0.3cm] at (Trd.south-|Trd.west) (dev) {train};
    \node[anchor=north west, standard, text width=2cm, fill=orange!30!white, right=0.5em of dev](eval) {eval};
    \node[anchor=north west, standard, text width=4cm, fill=red!30!white, yshift=-0.3cm] at (Ted.south-|Ted.west) (Ted2) {test};
    \end{tikzpicture}
\caption{Overview of the dataset split for the DCASE dataset}
\label{fig:DCASE-split}
\end{figure}
\noindent
The evaluation of the embedding space for the \fullref{sub:DCASE-Task-Dataset} is done in a few separate steps. The development dataset of the challenge is further split into an \texttt{development-training} and an \texttt{development-evaluation} set. First, an arbitrary embedding architecture is being trained on the \texttt{development-training} dataset and evaluated on the \texttt{development-evaluation} set using the metrics available for the embedding (\ref{sec:Metrics}), this process is repeated until an architecture with the desired performance is found.
\newline
\newline
To then further evaluate the performance of the resulting embedding space, a simple classifier is trained using the embeddings as input. There are two different classifiers used to evaluate the embedding space, a linear logistic classifier and a two-layered classifier. Both of the classifiers are visualised in figure \ref{fig:Classifier-DCASE-Visualisation}.
\newline
\newline
The linear logistic classifier (\ref{fig:logistic-classifier}) only consists of a single softmax output layer, which has the same amount of output units as there are classes in the dataset. The two-layered classifier (\ref{fig:dense-classifier}) consists of a two-layered dense model and a softmax output layer. Both classifiers are used to evaluate the embedding space. However, their idea is quite different from each other. The linear classifier tries to separate the embedding space using a hyperplane and therefore shows how well separated the embedding clusters are. The dense classifier shows how much of a performance gain there is when using the embedding space with a deeper \gls{NN}.
\newline
\newline
The macro-averaged F1 score of the linear logistic classifier is compared to other F1 scores from the experiment, to evaluate the different embedding spaces. The classifier is trained on the raw audio representation itself to provide a baseline F1 score, to show the performance gain when using the embedding space.
\newline
\newline
In the final step, the resulting embedding space, as well as the trained classifier will be used on the evaluation dataset of the challenge, where the resulting macro-averaged F1 score is compared to the accomplished score by the submitted models. The aim is to show the result which can be accomplished when using the dataset in an unsupervised setting and only focusing on the audio data itself rather than the label.
\newline
\newline
The overall goal is to achieve the highest possible macro-averaged F1 score on the evaluation dataset provided by the DCASE challenge. 
\newline
\newline
However, the main goal is to show that the resulting embedding space represents a specific structure of the dataset and therefore represents meaning. Since the embedding space only focuses on the audio stream itself, the optimal space clusters audio segments which sound similar in the nearby region irrelevant of the label. This evaluation is done manually by examining the embedding space.
\begin{figure}[htbp]
\centering
\begin{subfigure}{.5\linewidth}
  \centering
  \begin{tikzpicture}[start chain=going below, node distance=15pt,
        point/.append style={on chain, join=by {signal}},
        layer/.append style={on chain, join=by {signal}}]
        \node[point] {Input to classifier, embedded sample};
        \node[conv] {Dense layer (hidden layer 1): \\ 256 units, ReLU};
        \node[conv] {Dense layer (hidden layer 2): \\ 256 units, ReLU};
        \node[activation] {Dense output: \\ 9 units (num. of classes), softmax};
        \node[point] {Output};
    \end{tikzpicture}
  \caption{Dense classifier}
  \label{fig:dense-classifier}
\end{subfigure}%
\begin{subfigure}{.5\linewidth}
  \centering
  \begin{tikzpicture}[start chain=going below, node distance=15pt,
        point/.append style={on chain, join=by {signal}},
        layer/.append style={on chain, join=by {signal}}]
        \node[point] {Input to classifier, embedded sample};
        \node[activation] {Dense output: \\ 9 units (num. of classes), softmax};
        \node[point] {Output};
    \end{tikzpicture}
  \caption{Linear logistic classifier}
  \label{fig:logistic-classifier}
\end{subfigure}
\caption{Visualisation of the different classifier architectures}
\label{fig:Classifier-DCASE-Visualisation}
\end{figure}

\subsection{Music dataset}
\label{sub:Eval-Music}
The process of evaluating the embedding space for the music dataset (\ref{sub:Music-Dataset}) is a bit more complicated since there are no prior results on this exact dataset, nor a baseline model to compare it to. The first evaluation of the embedding is relatively similar to the one for the \fullref{sub:Eval-DCASE}. The dataset is split into an \texttt{training}, \texttt{evaluation} and \texttt{test} set. Then the \texttt{training} dataset is used to train the embedding architecture and is evaluated on the \texttt{evaluation} set with the metrics specified in metrics (\ref{sec:Metrics}). After that, the model, which resulted in the best metrics, is chosen and is further evaluated.
\newline
\newline
The idea of training a simple linear logistic classifier using the embedding space as input, which was specifically made for the comparison of the noise detection dataset, can however as well be used when evaluating the music dataset. The result of the classifier is then not used to compare the results to some other results, but to check how easy a linear classifier can separate the clusters, by trying to place a hyperplane between them. This result then gives some insights about the resulting clusters and their separation.
\newline
\newline
Since one of the requirements of the project is to examine the resulting clusters of the music embedding space, a simple clustering algorithm, such as \fullref{sub:K-Means}, is being applied on the resulting embeddings of the \texttt{test} set and afterwards on the full dataset. This should provide a crucial insight into the resulting clusters and the performance of the embedding architecture. It further aims to show which categories of the dataset are in the nearby region and therefore should represent similar categories. 
\newline
\newline
Another requirement of the embedding space is that it should provide some similarity measure for the embedded songs. Therefore the resulting clusters, as well as their distances between each other, are essential properties which will need to be evaluated. Since the examination of these distances as well as the similarity between each music genre is quite hard, mister Emanuel Oehri, who provided the music dataset (\ref{sub:Music-Dataset}), helps to examine the distances between the clusters and gives feedback about their similarity from his professional opinion. He will further examine the resulting similarity between two songs of different genres which are projected into their nearby region, as well as the similarity between segments which have an inconsistent neighbourhood. The evaluation with mister Emanuel Oehri will be structured as an interview, where the interview guide can be found in the appendix \ref{sec:Interview-Guide}. The results of the conducted qualitative analysis are then located in section \ref{sub:Results-Music-Qualitative-analysis}.

\section{Testing}
\label{sec:Testing}
All of the components in the project are tested with the TensorFlow unit tests module called \texttt{tf.test}\footnote{\url{https://www.tensorflow.org/api_docs/python/tf/test}}. A test environment is created, which contains a small fraction of the dataset to speed up the testing process because testing on the entire dataset would be infeasible. For the entire implementation process, the test first principle is used to produce better and more reliable code. All test cases are kept as small as possible. However, some require other dependencies, such as the input pipeline to work. Therefore some tests are quite big because they need a lot of different dependencies. However, since all of the components are tested as well, and the thesis is a research project, it is plausible.
\newline
\newline
Table \ref{tab:Components-Testing} lists all of the components in the project and gives detailed information about the testing method and the reason behind it. All of the unit tests can be found in the source repository of this project in the \texttt{test} directory, categorised by the component they test. Since some of the tests are conducted manually, a comprehensive test concept can be found in the appendix (\ref{tab:Test-Concept}) for the manual tests.
\begin{table}[htbp]
    \centering
    \caption{Testing method of each project component}
	\label{tab:Components-Testing}
    \begin{tabular}{p{.15\textwidth} | p{.20\textwidth} | p{.55\textwidth}}
        \toprule
        \textbf{Component} & \textbf{Testing method} & \textbf{Reason} \\ 
        \midrule[1pt]
        \texttt{dataset} & unit test & result can be validated, since result is deterministic \\
        \hline
        \texttt{feature\_extractor} & unit test & result can be validated, since result is deterministic \\
        \hline
        \texttt{input\_pipeline} & unit test & result can be validated, since result is deterministic \\
        \hline
        \texttt{loss}  & unit test & result can be validated, since result is deterministic \\
        \hline
        \texttt{models\_classifier} & unit test + manual testing & automated tests check if the model can be built and outputs values if the model trains can only be checked manually by observing the metrics \\
        \hline
        \texttt{models\_embedding} & unit test + manual testing & automated tests check if the model can be built and outputs values if the model trains can only be checked manually by observing the metrics \\
        \hline
        \texttt{training} & manual testing & result is not deterministic and can therefore only be tested manually by looking at the metrics over the training process \\
        \hline
        \texttt{utils} & unit test & result can be validated, since result is deterministic \\
        \bottomrule
    \end{tabular}
\end{table}

\section{Project structure}
\label{sec:Project-Structure}
The thesis consists of two projects, a project for the documentation and a project for the source of the thesis. Both projects are git repositories hosted on GitHub\footnote{\url{https://github.com/}}. For the duration of the thesis, both repositories are kept private, and after the completion of the project will be open-sourced.

\begin{figure}[ht]
    \dirtree{%
    .1 deep-embedded-music/.
    .2 data/.
    .2 experiments/.
    .3 config/.
    .3 DCASE/.
    .3 DJ/.
    .2 notebooks/.
    .2 src/.
    .2 test/.
    .2 test-environment/.
    .2 Dockerfile.
    .2 onstart.sh.
    .2 requirements.txt.
    }
\caption{Overview of the source repository \flqq deep embedded music\frqq}
\label{fig:Project-Overview-Source}
\end{figure}
\noindent
Figure \ref{fig:Project-Overview-Source} illustrates the project structure of the source repository, which can be found on GitHub\footnote{\url{https://github.com/FabianGroeger96/deep-embedded-music}}. The illustration shows an overview of the overall structure of the repository. More in-depth explanation can be found in section \ref{ch:Realisation}.

\begin{figure}[ht]
    \dirtree{%
    .1 baa-doc-deep-embedded-music/.
    .2 files/.
    .2 img/.
    .2 include/.
    .2 intermediate-presentation/.
    .2 final-presentation/.
    .2 study-doc/.
    .2 BAA-Documentation.tex.
    .2 README.md.
    .2 references.bib.
    }
\caption{Overview of the documentation repository \flqq baa-doc-deep-embedded-music\frqq}
\label{fig:Project-Overview-Documentation}
\end{figure}
\noindent
Figure \ref{fig:Project-Overview-Documentation} illustrates the project structure of the repository used for the documentation, which can be found on GitHub\footnote{\url{https://github.com/FabianGroeger96/baa-doc-deep-embedded-music}}. The repository contains a folder for the \textit{documentation}, \textit{interim presentation}, \textit{final presentation} and the \textit{study doc}. The study doc contains detailed information about the conducted experiments, which are attached in the appendix study doc (\ref{app:Study-Doc}).
\newline
\newline
Both datasets used for the thesis were used as they are and therefore did not need to be kept within a specific repository for the project. The dataset of the DCASE challenge 2018 - Task 5 consists of a development and evaluation dataset, which are both hosted on Zendo\footnote{\url{https://zenodo.org}}. The development dataset is available under \url{https://zenodo.org/record/1247102} while the evaluation dataset is available under \url{https://zenodo.org/record/1964758}. 
\newline
\newline
The music dataset was kindly provided by Mr Emanuel Oehri and is available as a private GitLab Git LFS repository, which will be kept as a private repository since all the songs provided are property of Mr Oehri.

\section{Resources}
\label{sec:Resources}
Deep neural networks architectures are computationally expensive what results in long-running experiments. Using powerful \glspl{GPU} reduces the processing time up to a factor of 20. The Lucerne University of Applied Science and Arts supported this thesis with a GPU GTX 1080Ti, 11 GB RAM, hosted within the enterprise lab\footnote{\url{https://www.enterpriselab.ch/}}. 
\newline
\newline
Additionally, \glspl{GPU} were rented on vast.ai\footnote{\url{https://vast.ai/}} during specific experiments to decrease the training time, by running experiments in parallel. A Tesla V100 \gls{GPU} 16 GB RAM was mostly rented to mitigate the problem of not having enough RAM on the GPU for experiments with large models (e.g. such as ResNet50).



\chapter{Realisation}
\label{ch:Realisation}
% TODO Beschreibung der Umsetzung der definierten Ziele, einschliesslich der aufgetretenen Schwierigkeiten und Einschränkungen

\begin{figure}
    \dirtree{%
    .1 src.
    .2 feature-extractor \ldots{} (audio representations).
    .2 input-pipeline \ldots{} (triplet input pipeline).
    .2 loss \ldots{} (implementation of loss functions).
    .2 models \ldots{} (implementation of models).
    .2 training \ldots{} (training utility functions).
    .2 utils \ldots{} (contains various utility functions).
    .2 trace-training.py \ldots{} (trace the training procedure).
    .2 train-classifier.py \ldots{} (training procedure for classifier).
    .2 train-triplet-loss.py \ldots{} (training procedure of triplet loss).
    }
\caption{Overview of the source structure of \flqq deep embedded music\frqq \ repository}
\label{fig:Overview-Source-Structure}
\end{figure}

\section{Data set}
\label{sec:Data-Set}

\subsection{Data set cleaning}
\label{sub:Data-Set-Cleaning}

\subsection{Statistical analysis of the data set}
\label{sub:Statistical-Analysis-Data-Set}

\section{Training}
\label{sec:Training}

\section{Prototype}
\label{sec:Prototype}


\chapter{Evaluation and validation}
\label{ch:Evalutation-Validation}

\section{DCASE 2018 challenge - task 5 dataset}
\label{sec:Results-DCASE}
This section describes the various experiments done on the \nameref{sub:DCASE-Task-Dataset}, where each led to a crucial conclusion for training the embedding space. These conclusions then led to the final hyperparameters used for training the embedding space for the DCASE dataset. This section further examines the embedding space (\ref{sub:Eval-Embedding-Space-DCASE}) and compares the logistic classifier trained on top of the embedding space with the results from the DCASE challenge (\ref{sub:Eval-Comparison-DCASE}). In the end, a conclusion of the experiments with the \nameref{sub:DCASE-Task-Dataset} is made (\ref{sub:Eval-Conclusion-DCASE}), which contains ideas on more experiments to conduct and further improvements.
\newline
\newline
For each experiment a detailed study doc is written, which includes in depth information about the conducted experiment. All of the study docs are attached as PDF, in the appendix \ref{app:Study-Doc}.

\subsection{Experiment: margin}
\begin{figure}[tb]
\centering
\begin{subfigure}{.5\linewidth}
  \centering
  \includegraphics[width=.9\linewidth]{study-doc/experiment_margin/assets/margin_all_plot.png}
  \caption{triplet loss from all margins}
  \label{fig:sub-margin-all}
\end{subfigure}%
\begin{subfigure}{.5\linewidth}
  \centering
  \includegraphics[width=.9\linewidth]{study-doc/experiment_margin/assets/margin_5_7_10_plot.png}
  \caption{triplet loss from margins=\texttt{0.5, 0.7, 1.0}}
  \label{fig:sub-margin-5-7-10}
\end{subfigure}
\caption{Plot of the triplet loss value of the margin experiment}
\label{fig:margin-experiment-triplet-loss-plot}
\end{figure}
\noindent
This experiment aims to show the affect of changing the margin $\alpha$ within a triplet loss on the DCASE dataset. A very important hyperparameter when training a triplet loss is the margin, denoted as $\alpha$. The margin makes sure that the network is not allowed to output the trivial solution, where all the embeddings vectors are zero or contain the same values. Within the triplet loss function, it is used to put a limit on how far the network can push the negative sample away to improve the loss. Thus the distance of the negative sample has to be higher than the distance from the anchor to the positive sample plus the margin $\alpha$. This experiment aims to show the importance of the margin as well as to find the optimal one for the DCASE dataset.
\newline
\newline
The hyperparameters used for this experiment are shown in the corresponding study doc in the appendix \ref{app:Study-Doc}. The margin is evaluated using a state of the art ResNet18 architecture on the DCASE dataset. The hyperparameters in section \textit{Feature representation} as well as the sample rate are the default ones proposed by the organisers of the DCASE challenge within the baseline project. The margin is evaluated for six different values \texttt{[0.3, 0.5, 0.7, 1, 2, 10]}.
\newline
\newline
Five models with the same hyperparameters, were trained for ten epochs. The value of the triplet loss is the most important criteria for selecting the optimal margin since the margin has the most significant impact on the loss function. Figure \ref{fig:sub-margin-all} shows all the trained models in a single plot to visualise the impact on changing the margin. From this figure, the resulting embeddings improved from the trivial solution as the margin is increased. Thus as the margin is decreased the total number of triplets generated whose loss is higher than zero decreases, therefore, they do not contribute to the training of the model thus reducing the accuracy of the outputted embedding’s.
\newline
\newline
There is a vast difference in the value of the loss values. The \texttt{margin=10} has by far the highest loss value of approximately 8, which is very intuitive since the distance to the opposite has to be higher than the distance to the neighbour plus the margin, which is in this case \texttt{8}. This constraint is tough to satisfy, and therefore the loss is relatively high.
\newline
\newline
The smallest loss value is the one from the \texttt{margin=0.3}. The reason for that is the same as for the high margin, but vice-versa. The constraint it has to satisfy is that the distance to the opposite has to be higher than the distance to the neighbour plus the margin. This is very easily satisfied since it is a relatively small distance.
\newline
\newline
The figure \ref{fig:sub-margin-5-7-10} shows the triplet loss of the margin values \texttt{0.5, 0.7, 1.0}. It can be seen that the loss is fairly similar and approaches zero.
\newline
\newline
Since the highest and lowest margin can be omitted simply by examining the triplet loss plot because the lowest margin does not contribute to the training and the highest sets a constraint which can only be satisfied in a small number of cases, in the other cases it is a much harder decision since the loss value is fairly the same. However, when looking at the resulting embedding space, it can be seen that the \texttt{margin=1.0} distances between the centroids of each label are fairly equally distributed, which is the optimal outcome. Figure \ref{fig:dist-margin-1} shows the distance matrix of the \texttt{margin=1.0}. Whereas the other margins do not have such an equally distributed distance matrix, which indicates that one or more label can be distinguished better than others. However, the optimal solution should be that the distances from all clusters are fairly the same. Therefore the \texttt{margin=1.0} is chosen as the optimal parameter for the DCASE dataset and is used in all of the further experiments as the standard hyperparameter.
\begin{figure}[tb]
\centering
    \includegraphics[width=0.5\linewidth]{study-doc/experiment_margin/assets/distance_mat_margin_1.png}
    \caption{Distance matrix of the margin=\texttt{1.0}}
    \label{fig:dist-margin-1}
\end{figure}

\subsection{Experiment: segment size}
\begin{figure}[tb]
\centering
    \includegraphics[width=0.5\linewidth]{study-doc/experiment_tile_size/assets/tile_sizes_plot.png}
    \caption{Plot of the triplet loss of the different tile sizes}
    \label{fig:tile-size-plot}
\end{figure}
\noindent
The experiment aims to show the effect of the size of the audio segments which is used as the size of the triplets. The size of each triplet, which is fed to the network, is an essential hyperparameter which needs to be carefully chosen. Because it specifies how much information each segment contains and is therefore fed to the network. If the segments are chosen too small, it does not contain enough information to distinguish between categories. If the size is too big, the segment contains too much information, and therefore, the model needs to work with a lot more data and gets a lot heavier. This experiment is conducted to search an optimal segment size for the triplets.
\newline
\newline
The hyperparameters used for this experiment are shown in the corresponding study doc in the appendix \ref{app:Study-Doc}. The experiment is conducted using a state of the art ResNet18 architecture on the DCASE dataset. The hyperparameters in section \textit{Feature representation} as well as the sample rate are the default ones proposed by the organisers of the DCASE challenge within the baseline project. The sample tile size is evaluated for three different values \texttt{[1, 2, 4]} in seconds.
\newline
\newline
Three models with the same hyperparameters were trained for 20 epochs. The value of the triplet loss is the primary evaluation criteria which is used to compare the different triplet sizes since it has the most effect on this value because it should show how much of an audio sample is needed to distinguish between different classes. Figure \ref{fig:tile-size-plot} shows all the trained models in a single plot to visualise the impact of changing the sample size. Nevertheless, it is quite hard to interpret the different graphs since all of them are near zero.
\newline
\newline
The figure \ref{fig:tile-size-plot} shows that the tile size 5s has the highest loss value while the tile sizes 1s and 2s have relatively similar values. However, this does not mean that the tile size of 5s is the worst out of the three, it instead means that the model finds it harder to distinguish audio files when a larger sample is available, which is pretty evident since longer samples contain more information.
\newline
\newline
The figure \ref{fig:tile-size-plot} further shows that the tile sizes of 1s and 2s have a very steep graph at the beginning and then hardly change their value. This indicates that it is relatively simple to achieve a good loss value with small audio samples, which means that the model can easily distinguish between small samples.
\newline
\newline
Both of these interpretations of the plot is pretty straight forward, but when the resulting embedding space is further examined using the Embedding Projector from the tensorboard, it can be seen that the tile size of 5s (\ref{fig:embedding-5s}) results in much clearer clusters than the tile size of 1s (\ref{fig:embedding-1s}). 
\newline
\newline
If the optimal parameter for the sample tile is only chosen from the loss value and therefore, from the plot \ref{fig:tile-size-plot}, it would be quite hard. However, since the visualisation of the embedding space shows a clear benefit in using a larger sample, the \textbf{sample tile size of 5s} is chosen to be the optimal one for the DCASE dataset. This can be explained because smaller samples much often contain sounds which do not indicate a specific sound in that class, say for example there is a two-second silence in a sound file of the class eating, it would be projected in the nearby region of silence in a sound of a different class, which is useful for other applications but since the goal is to achieve a best possible embedding space, this is not a satisfying result. Therefore, the larger sound segments are more robust to such problems, since they hold much more information about the resulting class. 
\newline
\newline
If the thesis focused on supervised triplet loss, it would make sense to cut the audio files in much smaller segments than in the unsupervised setting, since in supervised learning the triplet selection makes sure that the clustering focuses on the classes and not some other arbitrary criteria, which happens in unsupervised learning. In the unsupervised triplet loss, it is challenging to examine what exactly is being clustered, because there can be an underlying structure which can not be seen for us humans.
\begin{figure}[tb]
\centering
\begin{subfigure}{.5\linewidth}
  \centering
  \includegraphics[width=.9\linewidth]{study-doc/experiment_tile_size/assets/embedding_space_5s.png}
  \caption{tile size 5s}
  \label{fig:embedding-5s}
\end{subfigure}%
\begin{subfigure}{.5\linewidth}
  \centering
  \includegraphics[width=.9\linewidth]{study-doc/experiment_tile_size/assets/embedding_space_1s.png}
  \caption{tile size 1s}
  \label{fig:embedding-1s}
\end{subfigure}
\caption{Visualisation of the embedding space from the tile size 5s and 1s}
\label{fig:tile-size-experiment-embedding-space}
\end{figure}

\subsection{Experiment: embedding size}
\begin{figure}[t]
\centering
    \includegraphics[width=0.5\linewidth]{study-doc/experiment_embedding_size/assets/plot_embedding_sizes.png}
    \caption{Plot of the triplet loss of the different embedding sizes}
    \label{fig:plot-embeddings-epochs}
\end{figure}
The experiment aims to show the effect of the size of the last dense layer from the embedding architecture, further called the \textbf{embedding size}. The size of the embedding space is essential for the performance, since choosing the wrong hyperparameter can lead to over- or underfitting of the model. The size defines how many dimensions the resulting embedding space has. Therefore if this parameter $e$ is selected to be too big, the model almost certainly overfits, because the model has many options to project the input data onto the embedding space. However, if $e$ is chosen to be too small, there is not enough room to project inputs in different regions. This experiment aims to search an optimal parameter for $e$.
\newline
\newline
The experiment is conducted using a state of the art ResNet18 architecture on the DCASE dataset. The hyperparameters in section \textit{Feature representation} as well as the sample rate are the default ones proposed by the organisers of the DCASE challenge within the baseline project. The embedding size $e$ is evaluated for four different values \texttt{[2, 16, 32, 64]}.
\newline
\newline
Four models with the same hyperparameters were trained for a different amount of epochs. The training was stopped when no more learning was observed. 
\newline
\newline
Comparing different embedding sizes is pretty hard since most of the metrics in the thesis focus on distances between embedding points. In higher dimensional embedding spaces, distances have a different scale and different meanings. This is especially true if small embedding sizes, such as \texttt{2}, and large sizes, such as \texttt{64}, are compared with each other. Therefore a simple classifier was trained on the resulting embedding spaces, and the metrics of the classifier was compared to find the optimal parameter. To further compare the embedding spaces, they were visualised using the Tensorboard Embedding Projector and manually compared with each other.
\newline
\newline
The figure \ref{fig:plot-embeddings-epochs} shows the different triplet loss values of the embedding sizes. The embedding size \texttt{2} has a significantly higher value than other embedding sizes, which shows that in the two-dimensional embedding space, it is a lot harder to project the data points apart from each other. Whereas in high dimensional embedding spaces, the model can easier build clusters.
\begin{figure}[t]
\centering
\begin{subfigure}{.33\linewidth}
  \centering
  \includegraphics[width=.9\linewidth]{study-doc/experiment_embedding_size/assets/embedding_space_16.png}
  \caption{embedding size 16}
  \label{fig:embedding-space-16}
\end{subfigure}%
\begin{subfigure}{.33\linewidth}
  \centering
  \includegraphics[width=.9\linewidth]{study-doc/experiment_embedding_size/assets/embedding_space_32.png}
  \caption{embedding size 32}
  \label{fig:embedding-space-32}
\end{subfigure}%
\begin{subfigure}{.33\linewidth}
  \centering
  \includegraphics[width=.9\linewidth]{study-doc/experiment_embedding_size/assets/embedding_space_64.png}
  \caption{embedding size 64}
  \label{fig:embedding-space-64}
\end{subfigure}
\caption{Plot of the resulting embedding spaces}
\label{fig:embedding-size-experiment-embedding-space}
\end{figure}
\newline
\newline
\noindent
The plot further shows that the loss value of the embedding sizes \texttt{16, 32, 64} are relatively similar and are therefore further compared by examining their resulting embedding space. Which is shown in figure \ref{fig:embedding-space-16}, \ref{fig:embedding-space-32} and \ref{fig:embedding-space-64}. The result shows that there are vast differences in the embedding spaces, even though the triplet loss value is not that different. The embedding size of \texttt{16} shows only approximately four resulting clusters, which indicates a noisy embedding space where small classes are not well separated from each other. The embedding space of \texttt{32} and \texttt{64} show significant more resulting clusters and they result therefore in a better embedding space. However, it is to say that both embedding spaces further have more noise in it than the lower-dimensional space.
\newline
\newline
The line plot \ref{fig:embedding-size-classifier-f1} shows the resulting F1 score when a simple logistic classifier is trained on top of the resulting embedding space. Since the DCASE dataset is heavily unbalanced, the F1 score is compared. All of the classifiers are trained for 20 epochs using the same parameters as the one for training the embedding space. The figure \ref{fig:embedding-size-classifier-f1} shows that the F1 score of the embedding space \texttt{16} is the highest out of the three.
\newline
\newline
The line plot \ref{fig:embedding-size-classifier-loss} shows the sparse categorical cross-entropy loss value of the embedding spaces \texttt{16} and \texttt{64}.
\newline
\newline
The figure \ref{fig:plot-embeddings-epochs} shows that the smallest embedding size can be omitted since it has the highest triplet loss value significantly. The other three embedding spaces have quite similar values and are, therefore, further compared. The trained classifier on top of the embedding space shows (figure \ref{fig:embedding-size-classifier-f1}) that the embedding size \texttt{32} can be omitted since it has a significantly lower score than the others. The figure \ref{fig:embedding-size-classifier-f1} shows, that the embedding size \texttt{16} has achieved the highest F1 score of approximately \texttt{0.39}. However, the figure \ref{fig:embedding-size-classifier-loss} shows that the loss value of the embedding size \texttt{16} converges, which indicates that the training process is finished and the classifier will not show any improvements when training longer. The embedding size \texttt{64} has a lower F1 score, but the loss value is still decreasing at the end of epoch 20, which indicates that the model can benefit from further training. Further training will increase the F1 score until it converges.
\begin{figure}[t]
\captionsetup{format=plain}
\centering
\begin{subfigure}{.5\linewidth}
  \centering
  \includegraphics[width=.9\linewidth]{study-doc/experiment_embedding_size/assets/classifier_f1.png}
  \caption{F1 score}
  \label{fig:embedding-size-classifier-f1}
\end{subfigure}%
\begin{subfigure}{.5\linewidth}
  \centering
  \includegraphics[width=.9\linewidth]{study-doc/experiment_embedding_size/assets/classifier_loss.png}
  \caption{sparse categorical cross-entropy loss}
  \label{fig:embedding-size-classifier-loss}
\end{subfigure}
\caption{Line plot of the metrics from the different classifiers trained on top of the resulting embedding spaces}
\label{fig:embedding-size-experiment-classifier-metrics}
\end{figure}
\newline
\newline
\noindent
Because of this result, the optimal embedding size, out of the four, is \texttt{64}, since it has an optimal triplet loss value, a high enough F1 score and the loss value still decreases after 20 epochs of training the classifier.
\newline
\newline
This experiment showed that changing the dimension of the embedding space results in significant different embedding spaces and therefore resulting clusters. The experiment should further be conducted for the music dataset because this parameter highly depends on the underlying structure of the data. 
\newline
\newline
For the next experiments, the embedding space size \texttt{64} is chosen. The embedding space should also be evaluated on significant higher spaces, such as \texttt{256} or \texttt{512}.
\newline
\newline
The experiment further showed another important conclusion, that the longer the embedding space is trained, the more the loss value oscillates which indicates that a learning rate decay should be used to reduce the learning rate over time (\ref{fig:plot-triplet-64}).
\begin{figure}[h]
\centering
    \includegraphics[width=0.5\linewidth]{study-doc/experiment_embedding_size/assets/plot_triplet_loss.png}
    \caption{Plot of the triplet loss of the embedding space 64}
    \label{fig:plot-triplet-64}
\end{figure}

\subsection{Experiment: regularisation factor}
\begin{figure}[ht]
\centering
    \includegraphics[width=0.5\linewidth]{study-doc/experiment_regularisation/assets/triplet_loss.png}
    \caption{Plot of the triplet loss values of the different regularisation factors $\lambda$}
    \label{fig:plot-triplet-loss}
\end{figure}
\noindent
The experiment aims to show how the embedding space will change for different regularisation factors. The purpose of regularisation is that it prevents models from overfitting, by penalising high weights. The regularisation is calculated for all layers weights separately and is then added to the standard loss function, in this case, the triplet loss. The combined loss value is then used to calculate the gradients and update the weights of the model. There are a lot of different regularisation techniques, which successfully reduce the chance of overfitting. In this experiment, the \texttt{L2-Regularisation} is used and evaluated with different values. The parameter which will control the value of the \texttt{L2-Regularisation} is denoted as $\lambda$.
\newline
\newline
The experiment will be conducted using a state of the art ResNet18 architecture on the DCASE dataset. The hyperparameters in section \textit{Feature representation} as well as the sample rate are the default ones proposed by the organisers of the DCASE challenge within the baseline project. The regularisation factor $\lambda$ will be evaluated for three different values \texttt{[1.0, 0.1, 0.01]}.
\newline
\newline
Comparing the effect of different regularisation factors on the embedding space is pretty hard, since comparing embedding spaces is not very straight forward and is a rather tricky task. This is mainly because of the fact, that to visualise the high dimensional embedding space in a way humans can perceive it, it has to be reduced to two or three dimensions. Therefore the original space can not be examined, and the visual representation is always an approximation of the space in a lower dimension. To still compare the embedding spaces and the effect of $\lambda$, a simple logistic classifier is trained on top of the resulting embedding spaces, which aims to show how well a simple classifier works with the embedding space. This will show how good the resulting embedding space is. The classifier is trained for 20 epochs with the same parameters as the embedding model.
\newline
\newline
Figure \ref{fig:plot-triplet-loss} shows the difference between the triplet loss values of the embedding model when changing the regularisation factor. It shows that the model with $\lambda = 1.0$ has the highest triplet loss value, which is obvious, since the bigger the regularisation factor is, the longer the model optimises the weights to satisfy the constraint of having very small weights. The lower $\lambda$, the faster the model optimises the triplet loss value. $\lambda = 0.01$ has the lowest triplet loss value.
\begin{figure}[tb]
\captionsetup{format=plain}
\centering
\begin{subfigure}{.5\linewidth}
  \centering
  \includegraphics[width=.9\linewidth]{study-doc/experiment_embedding_size/assets/classifier_f1.png}
  \caption{F1 score}
  \label{fig:regularisation-experiment-classifier-f1}
\end{subfigure}%
\begin{subfigure}{.5\linewidth}
  \centering
  \includegraphics[width=.9\linewidth]{study-doc/experiment_embedding_size/assets/classifier_loss.png}
  \caption{sparse categorical cross-entropy loss}
  \label{fig:regularisation-experiment-classifier-loss}
\end{subfigure}
\caption{Line plot of the F1 score and the loss from the different classifiers trained on top of the resulting embedding spaces}
\label{fig:regularisation-experiment-classifier-metrics}
\end{figure}
\newline
\newline
\noindent
Figure \ref{fig:regularisation-experiment-classifier-f1} shows the different F1 scores of the logistic classifier, which is trained on top of the embedding architecture. This provides an idea of how well the embedding space separates the classes and therefore gives a performance gain. The figure \ref{fig:regularisation-experiment-classifier-f1} shows, that the regularisation factor $\lambda = 0.01$ has the highest F1 score. This result is significant since this particular embedding space was only trained for 30 epochs, unlike the other models, who are trained for 50. 
\newline
\newline
The figure \ref{fig:regularisation-experiment-classifier-loss} shows the corresponding categorical cross-entropy loss values of the trained classifiers. It shows that the loss of $\lambda = 1.0$ converges before $\lambda = 0.1$ or $\lambda = 0.01$. The figure further shows, that the loss values of $\lambda = 0.1$ and $\lambda = 0.01$ are not converged and would benefit from a longer training.
\newline
\newline
Figure \ref{fig:regularisation-experiment-classifier-f1} shows that the highest regularisation factor $\lambda = 1.0$ can be omitted since it did not accomplish an equally good F1 score. This result is expectable since a high regularisation factor forces the embedding model to work with very small weights and can, therefore, lead to performance loss. 
\newline
\newline
The regularisation factor $\lambda = 0.1$ and $\lambda = 0.01$ only show very little difference in F1 performance as well as in triplet loss value. Never the less, the factor $\lambda = 0.01$ provides slightly better results than $\lambda = 0.1$. 
\newline
\newline
The triplet loss value and the loss value of the classifier of the regularisation factor $\lambda = 0.01$ is still decreasing at the end of the training, which indicates that the model can benefit from further training. Further training will increase the resulting embedding space and the F1 score of the classifier until it converges.
\newline
\newline
The figure \ref{fig:plot-triplet-loss} and \ref{fig:regularisation-experiment-classifier-f1} further indicate one very important conclusion, that the value of the triplet loss is directly related to how well the classifier performs, which is further an indication how well the created embedding space works. Therefore the assumption is, that the lower the triplet loss value, the better the classifier performs and the better the embedding space is. 
\newline
\newline
The experiment shows that the regularisation factor $\lambda = 0.01$ gives the best results out of the three since it has an optimal triplet loss value, a high enough F1 score and the loss value still decreases after 20 epochs of training the classifier.
\newline
\newline
The experiment showed that the regularisation factor $\lambda$ has an essential impact on the embedding space and should therefore not be chosen to be too big nor too small since this either leads to over- or underfitting. The optimal regularisation parameter $\lambda$ should be further examined in later and more extensive experiments with a lot more values for $\lambda$.

\subsection{Experiment: feature representation}
\begin{figure}[htb]
\captionsetup{format=plain}
\centering
\begin{subfigure}{.5\linewidth}
  \centering
  \includegraphics[width=.9\linewidth]{study-doc/experiment_feature/assets/f1_feature_representation.png}
  \caption{Triplet loss}
  \label{fig:plot-triplet-loss-feature-representations}
\end{subfigure}%
\begin{subfigure}{.5\linewidth}
  \centering
  \includegraphics[width=.9\linewidth]{study-doc/experiment_embedding_size/assets/classifier_loss.png}
  \caption{F1 score of the classifier}
  \label{fig:classifier-f1-feature-represenations}
\end{subfigure}
\caption{Plot of the metrics from the models trained using the different feature representations}
\label{fig:feature-experiment-metrics}
\end{figure}
\noindent
The purpose of the feature representation is to represent the audio in a more compact form than the raw audio. The feature representation further determines the input size for the model. There are a lot of different ways to represent the an audio file in a more compact form. One of the most popular representations is the MFCCs, which is heavily used in the audio domain. However in the recent years, the trend leads more towards using the log Mel spectrogram, which is very similar to the MFCCs, but by omitting the last step of the calculation. This experiment aims to find the optimal feature representation for the thesis.
\newline
\newline
The experiment will be conducted using a state of the art ResNet18 architecture on the DCASE dataset. The hyperparameters in section \textit{Feature representation} as well as the sample rate are the default ones proposed by the organisers of the DCASE challenge within the baseline project.
The optimal feature representation will be evaluated for \texttt{[LogMel, MFCCs]}.
\newline
\newline
Comparing the effect of different regularisation factors on the embedding space is pretty hard, since comparing embedding spaces is not very straight forward and is a rather tricky task. This is mainly because of the fact, that to visualise the high dimensional embedding space in a way humans can perceive it, it has to be reduced to two or three dimensions. Therefore the original space can not be examined, and the visual representation is always an approximation of the space in a lower dimension. To still compare the feature representations, a simple logistic classifier is trained on top of the resulting embedding spaces, which aims to show how well a simple classifier works with the embedding space. This will show how good the resulting embedding space is. The classifier is trained for 40 epochs with the same parameters as the embedding model.
\newline
\newline
Figure \ref{fig:plot-triplet-loss-feature-representations} shows the difference between the triplet loss from the model using the different feature representations. It shows that the model trained using the MFCCs results in a significantly lower loss than the model using the log Mel spectrogram. The model using the log Mel spectrogram seem already converging at epoch 30.
\newline
\newline
Figure \ref{fig:classifier-f1-feature-represenations} shows the different F1 scores of the logistic classifier, which is trained using the embedding space as input. This provides an idea of how well the embedding space separates the classes and therefore gives a performance gain. The figure \ref{fig:classifier-f1-feature-represenations} shows clearly that the classifier trained on top of the log Mel spectrogram reached a higher F1 score. The metric further shows, that the classifier using the MFCCs embedding space fails to improve the F1 score over time, whereas the classifier for the log Mel spectrogram shows a definite increase over time.
\newline
\newline
From figure \ref{fig:plot-triplet-loss-feature-representations} it seems that the optimal feature representation is the MFCCs since it reached a significantly lower loss value. It further shows that the model would be able to benefit from further training since the model is still decreasing. However, when looking at the resulting F1 score of the classifiers (figure \ref{fig:classifier-f1-feature-represenations}), the result shows, that even though the MFCCs representation reached a lower triplet loss, the classifier fails to separate the resulting clusters using a hyperplane. 
\newline
\newline
This experiment shows that the optimal feature representation for the current thesis is using the log Mel spectrogram, since it resulted in a higher F1 score of the classifier, even though the triplet loss value is higher than the one using the MFCCs. 
\newline
\newline
This is mainly due to the different nature of the feature representations. The MFCCs representation input size (498, 13) is significantly lower than the log Mel spectrogram representation input size (498, 128). The MFCCs is a more compact representation, which seems to have a negative effect on the model's performance.
\newline
\newline
This experiment shows that the model benefits from using a representation which has more features and therefore, the log Mel spectrogram is used as the optimal representation for the thesis.

\subsection{Triplet loss}
The first experiments were conducted using the standard triplet loss equation given by \ref{eq:Triplet-Loss}. The loss value is given by calculating the triplet loss for the entire batch of triplets and afterwards taking the mean of the batch of triplet losses. This resulted in the triplet loss value, which was used to optimise the model. After a few experiments, one could identify that the triplet loss value decreased rapidly to almost zero and then only oscillated minimally. A low loss value typically means that the model is performing well, and the weights of the network only need to be changed slightly, and therefore the training is almost finished. However, it was observed that this was not true for the trained models since there was still much noise in the embedding space. This indicated that the loss value did not represent the actual performance of the model accurately.
\newline
\newline
In order to solve this problem, the calculation of the loss was further examined. This showed that after the first few epochs of training, many triplets did already satisfy the equation \ref{eq:Triplet-Loss}. However, this is a normal behaviour when training an unsupervised triplet loss, since the triplet selection is purely based on assumptions and not facts like it is for supervised triplet loss. When using assumptions, there is a high possibility, that the triplets are easily satisfied and therefore classify as easy triplets. It would be optimal always to select the hardest neighbour as well as selecting the hardest opposite for each anchor segment. However, this is not feasible and is, therefore neglected.
\newline
\newline
When calculating the triplet loss of a batch of triplets where 70\% of the batch satisfy the triplet loss constraint, the batch of triplet losses contains 70\% zeros. Afterwards, when taking the mean of the batch, the resulting loss value is really low, only because it consists of many zeros. Therefore the loss function is optimised to filter the zero values and only to calculate the mean of the non-zero loss values. This results in a significantly higher loss value, even in later epochs. The equation \ref{eq:Triplet-loss-justification} shows the difference between the ordinary and the mean filtered loss value, for illustration purposes, a batch size of 10 is taken.
\myequations{Justification behind triplet loss changes}
\begin{equation}
    \centering
    \begin{gathered}
        \text{triplet loss with mean:}\\
        \mathcal{L} = \frac{0 + 0+ 0+ 0+ 0+ 0+ 0+ 0.3+ 0.6+ 0.8}{10} = 0.17 \\
        \\
        \text{triplet loss with mean filtering:}\\
        \mathcal{L} = \frac{0+ 0+ 0+ 0+ 0+ 0+ 0+ 0.3+ 0.6+ 0.8}{3} = 0.56
    \end{gathered}
    \label{eq:Triplet-loss-justification}
\end{equation}
Two embedding models were trained using the same hyperparameters for 40 epochs, to test the newly proposed triplet loss calculation. The only difference is the calculation of the triplet loss. After the training has finished, a logistic classifier was trained for 20 epochs using the resulted embedding space. Both classifiers were then compared with each other, using the F1 score. This comparison is shown in figure \ref{fig:Triplet-Loss-Techniques}. There is a significant difference between the resulting scores, which indicates that the optimisation proposed above provided a significant benefit in comparison to the triplet loss without filtering. This notable difference is mainly because when training a model with mean filtering, the loss value remains a lot higher than without it, which results in better optimisation of the model. The mean filtering triplet loss is therefore used for all the experiments, because it provided a significant benefit.
\begin{figure}[ht]
\centering
    \includegraphics[width=0.6\linewidth]{img/Triplet-Loss-Comparison-Techniques.png}
    \caption{Triplet loss comparison of different techniques}
    \label{fig:Triplet-Loss-Techniques}
\end{figure}

\subsection{Embedding space}
\label{sub:Eval-Embedding-Space-DCASE}
\begin{table}[ht]
    \captionsetup{format=plain}
    \centering
    \caption{Hyperparameters of the optimal embedding architecture for the noise detection dataset}
	\label{tab:Hyperparameters-DCASE}
    \begin{tabular}{l|l}
        \toprule
        \textbf{Hyperparameter} & \textbf{value} \\ 
        \midrule[1pt]
        Model & ResNet18 \\ 
        \hline
        Epochs & 110 \\ 
        \hline
        Batch size & 64 \\ 
        \hline
        Optimizer & Adam \\ 
        \hline
        Learning rate & 1e-5 \\
        \hline
        Margin & 1.0 \\
        \hline
        L2 regularisation amount & 0.01 \\
        \hline
        Embedding dimension & 256 \\
        \midrule[1pt]
        \multicolumn{2}{l}{\textit{Audio sample}} \\
        \midrule[1pt]
        Sample rate & 16000 \\ 
        \hline
        Sample tile size & 5 \\
        \hline
        Sample tile range & 5 \\
        \hline
        Convert to mono & True \\
        \midrule[1pt]
        \multicolumn{2}{l}{\textit{Feature representation}} \\
        \midrule[1pt]
        Feature extractor & LogMelExtractor \\ 
        \hline
        Input size & 498 x 128 \\
        \bottomrule
    \end{tabular}
\end{table}
\begin{figure}[ht]
\centering
    \includegraphics[width=0.6\linewidth]{img/Triplet_loss_DCASE_final.png}
    \caption{Triplet loss plot of the final embedding model for the noise detection dataset}
    \label{fig:Triplet-Loss-DCASE}
\end{figure}
The optimal embedding space architecture for the DCASE task 5 challenge in this thesis is given by the hyperparameters in table \ref{tab:Hyperparameters-DCASE} as well as in table \ref{tab:Hyperparameters-Detailed-DCASE}, which shows the hyperparameters in more detail.
\newline
\newline
The model was trained for 110 epochs and reached a triplet loss value of 0.1879, which is by far the lowest loss of all the different experiments. The triplet loss value is shown in figure \ref{fig:Triplet-Loss-DCASE}. The plot shows that the model has decreased a significant amount between the 34 and the 39 epoch. 
\newline
\newline
The embedding space was further examined to get detailed insights about the resulting clusters and the distances between them.  The entire development dataset was projected onto the embedding space and saved, including with the corresponding label, name and segment, to examine the embedding space. Afterwards the embedding space was converted into an \texttt{cKDTree} from \texttt{scipy}, which is a kd-tree for quick nearest-neighbor lookup. This tree was used to get the neighbours from a specific data point in the embedding space.
\newline
\newline
First, the miss-classified embeddings were further examined, this is done by looping over the entire embedding space and checking if the label of the selected embedding and 20 of its neighbours are all different. If that is so, the segment and its neighbours are further examined. The examination showed that most of the segments which belong to a different cluster as their label, consisted of a big part of silence or were purely silence, and were therefore projected to the region of silence. Most of these segments consisted of a part of silence, and after for example, seven seconds a sound was observed. However, when segmenting each audio in segments, there is a high possibility, that the segment only consists of silence. This shows that the embedding space clusters the segments by their meaning and not by their activity, which means that silence when performing, for example, the activity \textit{working} or \textit{eating}, will be projected in the near-by region. It was also observed that there are a lot of audios, which do not contain any sound and are therefore purely silence but have the label \textit{working}. These audio files should be examined further because there is a chance that they are miss-classified and should be assigned to the label \textit{absence}. However, it can be reasoned, that silence when performing the activity \textit{working} belongs to the activity, since there is a high possibility that the person does not make any sound but the model should still recognise this activity as \textit{working}. Nevertheless, this provided useful insights into the dataset as well as the embedding space. It was also observed that there are some audios, where a microphone malfunction is present and because of that are wrong classified, for example audio \texttt{DevNode3\_ex53\_42}.
\newline
\newline
The other examination of the embedding space focused on the correct classified segments and aimed to show the nature of the embedding space. The idea is to select two labels of the dataset, for example, \textit{working} and \textit{eating}, then to loop over the dataset and checking if the embedding belongs to one of these classes and further has five neighbours from the same label. If this is the case, the embedding will be appended to a list of the corresponding label. These lists are then used to calculate the mean of the clusters from each label. The checking of the neighbourhood of each embedding is used to neglect the problem of outliers in clusters. Afterwards, the distance between the clusters is calculated and divided by a specified number of steps, which indicates how many steps have to be taken to reach the other label. Then the centroid of the first label is taken, and the step difference is added to it. From this position, the nearest neighbours are taken and displayed. This process is repeated until the position of the other centroid label is reached. This \flqq walk through the embedding space\frqq can then be represented as an image (\ref{fig:Walk-through-DCASE}) or as a continuous recording of the segments appended with each other. This representation shows that the embedding space represents much meaning because when walking through the space, for example from \textit{working} to \textit{social\_activity} the figure \ref{fig:Walk-through-DCASE} shows that there is more sound activity when approaching \textit{social\_activity}. However this increase is done in a rather slow matter and for example in the figure \ref{fig:Walk-through-DCASE}, the line three represents audios from \textit{social\_activity} where there is still a lot of silence or quiet talking.
\begin{figure}[ht]
\captionsetup{format=plain}
\centering
    \includegraphics[width=0.8\linewidth]{img/Walk_through_dcase_space.png}
    \caption{\flqq walk through the embedding space\frqq \ of the DCASE space from \textit{working} to \textit{social\_activity}}
    \label{fig:Walk-through-DCASE}
\end{figure}

\subsection{Comparison to challenge results}
\label{sub:Eval-Comparison-DCASE}
One of the primary evaluations for the embedding space is how well a simple logistic classifier, which is trained on top of the embedding space, performs. The metrics of the classifier are used to compare and evaluate different embedding spaces and also to compare how well it performs in comparison to the models submitted in the DCASE task 5 challenge. The submitted architectures were compared using the F1 score of the evaluation dataset using. The main goal of the challenge was to show that models are able to recognise daily activities using different microphone arrays. Therefore, to compare the models in the challenge, the F1 score was calculated from the evaluation set where the microphones were unknown and were not present in the training dataset. The winner of the challenge was a team from IBM, which accomplished an F1 score on the unknown microphones of 88.4\% and a 90.4\% F1 score on the entire evaluation dataset. The embedding space in this thesis was only compared using the full evaluation dataset.
\newline
\newline
On the final embedding architecture for this challenge, described in \ref{sub:Eval-Embedding-Space-DCASE}, the trained classifier accomplished a macro-averaged F1 score of 62.19\% on the full evaluation dataset. The difference between the results of this thesis and the challenge is approximately 30\% to the winner and 20\% to the others. A detailed classification report is given by figure \ref{fig:classification-report-DCASE}, which shows the precision, recall and F1-score of each class. It can be seen that the classes \textit{other} and \textit{eating} are the hardest to classify. This is mainly due to the nature of their sounds and the high imbalance in the dataset. The class \textit{other} is used when an activity is performed, which does not belong to a label in the dataset and therefore contains a lot of different sounds. The label \textit{eating} is very hard to differentiate, because the sound of eating itself is very similar to \textit{dish washing} or \textit{cooking} since there are mainly noises of kitchen utilities present.
\begin{figure}[ht]
\captionsetup{format=plain}
\centering
    \includegraphics[width=0.8\linewidth]{img/DCASE_F1_classification_report.png}
    \caption{Classification report of the classifier on the evaluation dataset from the DCASE challenge}
    \label{fig:classification-report-DCASE}
\end{figure}

\subsection{Conclusion}
\label{sub:Eval-Conclusion-DCASE}
The results from the examination of the embedding space for the DCASE challenge are quite astonishing since the embedding space represents much meaning, such as the distribution of the clusters and miss-classified segments are off place. It is fascinating that all of this meaning can be extracted purely based on the nature of the sound rather than the underlying label. One of the most astonishing properties of the embedding space is that it can find miss-classified audios or even microphone malfunctions within the dataset. 
\newline
\newline
However, when comparing the model to the challenge results, the embedding space performs not so well, which means that there is a 20\% gap between the trained embedding model and the challenge results. Nevertheless, since the goal was not to beat the participants in the challenge, but rather to create an embedding space which represents meaning and the comparison to the challenge was only used to show the difference between the approaches, the accomplished results compared to the challenge are still good. Furthermore, if a more extensive architecture, such as a ResNet, would be trained on top of the embedding space, the results could be profoundly improved.
\newline
\newline
The resulting embedding model uses a state-of-the-art ResNet18 architecture, which resulted in good results. More massive architectures such as ResNet50 or ResNet152 were not evaluated due to the lack of time and resources. Other approaches, such as \gls{CRNN}, should also be evaluated to find the optimal state-of-the-art architecture. There is a possibility that using more massive architectures would result in a better embedding space since there are more weights available. However, the possibility of overfitting also increases. Nevertheless, experiments have to show how the embedding space varies when increasing the architecture and if there is a performance gain when doing so.

\section{Music dataset}
\label{sec:Results-Music}

\subsection{Embedding space}

\subsection{Clustering applied to embedding space}

\subsection{Qualitative analysis}
\label{sub:Qualitative-analysis}

\subsection{Conclusion}



\chapter{Conclusion}
\label{ch:Conclusion}
This chapter presents the conclusion of the project, which summarises all of the results from the section evaluation and validation before (\ref{sec:Project-Conclusion}), described the discussion of the thesis in comparison to other research (\ref{sec:Discussion}) and provides useful information about further improvements and an outlook of the thesis (\ref{sec:Outlook}).

\section{Project conclusion}
\label{sec:Project-Conclusion}
This thesis aimed to find an embedding space for both noise detection and music, where the distances in the embedding space represent the similarity between them. The embedding space should be trained using only unsupervised machine learning techniques and therefore should focus on the underlying structure of the audio, rather than its label. To create such an embedding space, Tile2Vec, an image embedding algorithm should be adapted to the audio domain. 
\newline
\newline
Based on the extensive research phase conducted at the start of the project, the concepts, which needed to be used during the thesis, were easily understood. It further provided a lot of insights into the current research in the audio domain more specifically in the unsupervised audio domain. During the research, exciting and useful ideas were found, which then led to the ideas on how to realise this thesis's problem definition. The project plan created at the beginning of the thesis was surprisingly accurate and could be fulfilled almost over the entire project. Further, the chosen linear procedure model, showed many benefits, since, throughout the project, clear deliverables were defined and had to be concluded before moving to the next phase. 
\newline
\newline
Since using TensorFlow 2.0 for the realisation of the thesis, it was surprisingly straightforward to implement state-of-the-art architectures and input pipelines. One of the greatest difficulties during the implementation was that the input pipeline since working with audio data was very slow when using a typical TensorFlow generator pattern, which further resulted in not complete utilisation of the computing resources. A multi-threaded generator had to be implemented from scratch, since TensorFlow does not provide such an implementation by the time of writing, to mitigate that problem. Another difficulty during the realisation was the implementation of the datasets, which included the triplet selection for both datasets, in a matter, that both datasets could be used similarly from the input pipeline and that they can be changed by modifying a single hyperparameter. During the realisation, the advantages of using the test first principle were shown many times. Due to using this principle, many problems, which were related to logical mistakes, could be mitigated and therefore led to better code throughout the entire implementation phase. However, it is to say that it is not always easy to implement useful test cases when working with neural networks since some tests can only be conducted manually.
\newline
\newline
During the experiment phase, the thoughts, which led to the current implementation showed its convenience and practicality, since to conduct a specific experiment only one file, the \texttt{params.json}, had to be changed and it could be started. This led to rapid experiments, however, drained by very long training times. During the vast amount of experiments conducted, many characteristics of the embedding space were shown and were further used to improve the resulting space. The experiments showed, which parameters had the most impact when training an embedding space using an unsupervised triplet loss in the audio domain. For example, it showed, that the larger the segment of the triplets, the better the triplet loss architecture performs, which further leads to a much clearer embedding space.
\newline
\newline
The results from the final embedding space for the \gls{DCASE} challenge dataset showed in comparison to the results of the challenge lower results. However, the results accomplished from the model were using the dataset as unsupervised and by purely focusing on the underlying structure of each audio segment. Thus, the results accomplished are still impressive. Further, it is to say, that these results could easily be higher when using a deeper architecture for the classifier, which is trained using the embedding space as input. However, since the main goal of the thesis was to provide an embedding space, which could represent similarities and dissimilarities between audios, the model showed its true application and advantages. The embedding space of the noise detection dataset succeeded in finding audio files, which were misclassified or contained a microphone malfunction, by purely looking at the neighbourhood of each audio segment. These results are awe-inspiring, because it shows, that a neural network can find structures in audios which sound reasonable, even without the underlying label.
\newline
\newline
The same architecture for training the noise detection embedding space was used to train the embedding space for music. Based on the examination of the embedding space and the qualitative analysis, the results showed even more significant results. It showed that the embedding space succeeded in separating the different sub-genres of the dataset, which is already quite astonishing since these genres are very entangled and the differences are as well for an expert hard to find. This was mainly shown when applying a simple clustering algorithm on the embedded dataset, which resulted in clear clusters of each of the seven categories. Further, the results of a simple classifier trained, using the embedded segment as input, showed, that even when using only a logistic classifier the clusters were easy separable such that an F1 score of approximately 84\% resulted. When examining the embedding space further with the help of an expert, the results showed, that the embedding space found useful similarities in song segments even if they did not correspond to the same sub-genre. The examination further showed that the embedding space was able to create transitions between the sub-genres in the dataset, which were very impressive and sounded interesting.
\newline
\newline
Due to the lack of time and the lack of computing resources, deeper and different neural network architectures were not able to be tested. Deeper state-of-the-art \gls{CNN} architectures should be evaluated as well as state-of-the-art \gls{CRNN} architectures. Further research has to show if using such larger architectures lead to a performance gain or not.
\newline
\newline
Overall the thesis and the results found are considered to be a great success since all of the requirements of the project were able to be satisfied. The embedding space showed even more useful properties than intentionally imagined. The embedding space showed more remarkable performance on the music dataset, even in such entangled sub-genes than on the \gls{DCASE} challenge dataset. However, at the beginning of the project, it was assumed that the results would come out the other way round and the embedding space would perform better on the noise detection dataset than on the music dataset. This was mainly because the use of the music dataset was considered to be a harder challenge.

\section{Discussion}
\label{sec:Discussion}
The results indicate that the embedding space succeeded in finding similarities between audio samples, which was confirmed by the qualitative analysis conducted on the resulting music embedding space. Results from the noise detection embedding space, such as that it found misclassified segments in the dataset, showed that it as well succeeded in finding similarities between the categories in the noise detection dataset. The successful application can be traced back to the differences in the categories of the audio because even in such entangled sub-genres as in the music dataset, there is a difference in the audio, which can be found and used to cluster similarities. These differences can be found, because of the use of triplet loss, which tries to minimise the distance between an anchor and a neighbour segment, while maximising the distance between an anchor and an opposite segment. When adapting this idea to an unsupervised setting and selecting the neighbour segment to belong to the same audio as the anchor and the neighbour to belong to a different audio, the algorithm gets very clear constraints to satisfy. These constraints indicate that the algorithm should project segments of the same song in the near-by region, which was able to be observed when examining both of the embedding spaces.
\newline
\newline
These results build on existing evidence of the distributional hypothesis from natural language, which states that words appearing in similar contexts tend to have similar meanings. The results from this thesis indicate that this hypothesis holds even in the audio domain. Which further indicates, that audio segments, which appear in similar contexts, such as consecutively segments, tend to have similar meanings. However, the results from the qualitative analysis showed that this hypothesis has limitations when looking at an entire song of a specific genre. The hypothesis only holds if enough spectral information is present in the audio, which further indicates that at the start or the end of an audio, certain miss-classifications can happen. Further research is needed to establish if using larger segments would mitigate that difficulty and result in a better performance of the embedding space. 

\section{Outlook}
\label{sec:Outlook}
The results of the thesis showed that it is possible to find similarities between audios of different kinds only by using the audio itself. However, a few flaws in the embedding space were observed, which need to be mitigated in the following research. Such as experiments with larger neural network architectures should be conducted, and see if there is a performance gain in using them. Further, longer audio segments should be used to train the embedding space to see if there is a performance gain. To further improve the performance of the embedding space, the implementation should feature the possibility to specify how many triplets should be generated from each anchor. This would result in a much larger dataset, which leads to a better performing embedding space.
\newline
\newline
This property of finding similarities and dissimilarities can be beneficial for various different applications, such as to find similar songs to the one given or to find similar songs segments to the one provided. It would be even possible to create transitions in between sub-genres, which can be generated by finding the smallest distance between the genres and then using the nearest songs of the distance. The embedding space can further be used to help labels or artists classify their songs by showing them the most similar song to theirs with its corresponding label. 


\newpage

\pagenumbering{Roman}

\appendix
% Verhindert das Einfügen von Kapiteltitel kleiner als \chapter
\addtocontents{toc}{\protect\setcounter{tocdepth}{0}}

\listoffigures

\listoftables

\listofmyequations 
\pagebreak

\printglossary[type=\acronymtype, title=List of Abbreviations]

\printbibliography[heading=bibintoc,title=Bibliography]

\chapter{Appendix}
\label{app:Appendix}

\section{Milestone reports}
\label{app:Milestone-Reports}
This chapter lists the individual milestone reports. These should provide information on the status of the project during the realisation. For every completed milestone, a report was written before the next phase could be started. The table \fullref{tab:Milestones} shows the milestones within the thesis as well as the corresponding dates when the milestone and the phase will be completed and reviewed. The milestones are also shown in the whole project plan in \fullref{fig:Project-Plan}.

\subsection{Milestone report M1 from the 15.03.2020}
In the first phase of the project, the main goal was to research the subjects used for the thesis. Within phase, the project also had to be initialised and started. More precisely, the project structure and documentation also had to be defined. The goal was also to finish the first chapter of the thesis \fullref{ch:Related-Work} along with the research.

\begin{table}[htbp]
    \centering
    \caption{Work carried out milestone M1 (15.03.2020)}
	\label{tab:Work-Carried-Out-M1}
    \begin{tabular}{p{.70\textwidth} | p{.20\textwidth}}
        \toprule
        \textbf{Task} & \textbf{Status} \\ 
        \midrule[1pt]
        Research: Dataset & finished \\
        \hline
        Research: Audio processing & finished \\
        \hline
        Research: Triplet Loss & finished \\
        \hline
        Research: Tile2Vec & finished \\
        \hline
        Research: Evaluate research & finished \\
        \hline
        Realisation: Project setup & started \\
        \hline
        Realisation: Input pipeline & started \\
        \bottomrule
    \end{tabular}
\end{table}

\subsubsection{What work was carried out in the last reporting period}
The tasks carried out within this period are shown in the table \fullref{tab:Work-Carried-Out-M1}, along with the status of the task at the time of the milestone review. A detailed report on the entire tasks carried out within the period is shown in the work journal \ref{tab:Work-Journal}. To reach the goal of simultaneously finishing the chapter \ref{ch:Related-Work} with the research, each researched topic was documented right away. 

\subsubsection{State of progress}
The current milestone was successfully reached since all the planned tasks could be finished. The phase was finished too early so that the next period could already be started, the realisation phase. 

\subsubsection{Top three risks including planned measures}
\begin{enumerate}
    \setlength\itemsep{0em}
    \item Formulas related work wrong, will be checked by Daniel Pfäffli
    \item Topics missing in related work, will be checked by Daniel Pfäffli
    \item Input pipeline structure, will be discussed at the next meeting
\end{enumerate}

\subsection{Milestone report M2 from the 29.03.2020}
In the second milestone of the project, the main goal was to finish the whole project setup. The project repository had to be set up, the input pipeline was created for the dataset, and the default model architecture was created. The purpose was that after this milestone, the project is at a certain point, that the implementation of specific architectures is fast enough to experiment with different ones.

\begin{table}[htbp]
    \centering
    \caption{Work carried out milestone M2 (29.03.2020)}
	\label{tab:Work-Carried-Out-M2}
    \begin{tabular}{p{.70\textwidth} | p{.20\textwidth}}
        \toprule
        \textbf{Task} & \textbf{Status} \\ 
        \midrule[1pt]
        Realisation: Project setup & finished \\
        \hline
        Realisation: Input pipeline & finished \\
        \hline
        Realisation: Default architecture & finished \\
        \hline
        Realisation: Evaluation Workflow & midway \\
        \hline
        Realisation: Unit tests & midway \\
        \bottomrule
    \end{tabular}
\end{table}

\subsubsection{What work was carried out in the last reporting period}
The tasks carried out within this period are shown in the table \fullref{tab:Work-Carried-Out-M2}, along with the status of the task at the time of the milestone review. A detailed report on the entire tasks carried out within the period is shown in the work journal \ref{tab:Work-Journal}. Two tasks were not completed within the period, which was now prioritised in the next phase and will get completed first. The milestone could not be completed because to little time was planned for tasks like \flqq input pipeline\frqq or \flqq evaluation workflow\frqq. The evaluation workflow was completed except for the classifier, to evaluate the architecture in relation to the other models in the DCASE Challenge. Unit tests were not written for all the created models but will be right away at the start of the next milestone.

\subsubsection{State of progress}
The current milestone was not reached since two tasks could not be finished entirely. The uncompleted tasks were prioritised and were moved to the next phase of the project.

\subsubsection{Top three risks including planned measures}
\begin{enumerate}
    \setlength\itemsep{0em}
    \item GPU caching / file deletion problem, will be discussed at the next meeting
    \item Classifier architecture, will be discussed at the next meeting
    \item Unit tests for models, will be discussed at the next meeting 
\end{enumerate}

\subsection{Milestone report M3 from the 12.04.2020}
The third milestone aimed to finalise the realisation. The idea of unsupervised triplet loss had to be implemented, like the one from Tile2Vec for both datasets. Furthermore, an easy to use architecture for the experiments had to be developed. Thus the experiments should be conducted relatively simple. The main goal is to provide an architecture which is reliable, arbitrarily expandable and repeatable. All of these points are essential for conducting successful experiments.

\begin{table}[htbp]
    \centering
    \caption{Work carried out milestone M3 (12.04.2020)}
	\label{tab:Work-Carried-Out-M3}
    \begin{tabular}{p{.70\textwidth} | p{.20\textwidth}}
        \toprule
        \textbf{Task} & \textbf{Status} \\ 
        \midrule[1pt]
        Realisation: Evaluation Workflow & finished \\
        \hline
        Realisation: Unit tests & finished \\
        \hline
        Realisation: Tile2Vec implementation & finished \\
        \hline
        Realisation: Architecture for experiments & finished \\
        \hline
        Experiments: Conduct experiments & started \\
        \bottomrule
    \end{tabular}
\end{table}

\subsubsection{What work was carried out in the last reporting period}
The tasks carried out within this period are shown in the table \fullref{tab:Work-Carried-Out-M3}, along with the status of the task at the time of the milestone review. A detailed report on the entire tasks carried out within the period is shown in the work journal \ref{tab:Work-Journal}. Two tasks were not completed during the last phase and were completed at the start of this phase by prioritising them. Both of the other tasks which were planned to complete during this phase were completed.

\subsubsection{State of progress}
The current milestone was successfully reached since all the planned tasks could be finished. The phase was finished too early so that the next period could already be started, the experiment phase. 

\subsubsection{Top three risks including planned measures}
\begin{enumerate}
    \setlength\itemsep{0em}
    \item Review evaluation workflow, will be discussed at the next meeting
    \item Review of study doc, will be discussed at the next meeting
    \item Review of test concept, will be discussed at the next meeting 
\end{enumerate}

\subsection{Milestone report M5 from the 03.05.2020}
The last milestone before the thesis submission is the M5, which concludes the experiment phase. In the experiment phase, all of the experiments are conducted and validated. For each experiment, a separate study doc is written, which describes the experiments in more detail, and can be found in \ref{app:Study-Doc}. Within this phase, a lot of different experiments are conducted to find the optimal parameters for the problem definition. In this phase, the codebase will be adjusted if problems are found or if new ideas need to be implemented. The purpose of these adjustments is always to improve the performance of the models.
\begin{table}[htbp]
    \centering
    \caption{Work carried out milestone M5 (03.05.2020)}
	\label{tab:Work-Carried-Out-M5}
    \begin{tabular}{p{.70\textwidth} | p{.20\textwidth}}
        \toprule
        \textbf{Task} & \textbf{Status} \\ 
        \midrule[1pt]
        Experiments: Conduct experiments & midway \\
        \hline
        Experiments: Validate experiments & midway \\
        \bottomrule
    \end{tabular}
\end{table}

\subsubsection{What work was carried out in the last reporting period}
The tasks carried out within this period are shown in the table \fullref{tab:Work-Carried-Out-M5}, along with the status of the task at the time of the milestone review. A detailed report on the entire tasks carried out within the period is shown in the work journal \ref{tab:Work-Journal}. Both of the tasks were not completed during the phase. That the tasks were not completed is because there were more experiments to conduct than intentionally thought. There was also the problem that some of the experiments took a long time to be conducted due to the lack of additional computing resources. Both tasks will be prioritised and completed in the next phase. 

\subsubsection{State of progress}
The current milestone was not reached since two tasks could not be finished entirely. The uncompleted tasks were prioritised and were moved to the next phase of the project.

\subsubsection{Top three risks including planned measures}
\begin{enumerate}
    \setlength\itemsep{0em}
    \item Not enough time to conduct all experiments; therefore the buffer phase will be used to conduct additional experiments
    \item Not enough time to conduct all experiments; therefore additional computing power will be rented at \texttt{vast.ai}
    \item Not enough experience with the hyperparameters; therefore will be discussed within the next meeting
\end{enumerate}

\clearpage
\section{SINS database table}
\label{app:SINS-Database-Table}
This chapter shows a table with the different activities recorded within the SINS database. The activities are divided into the room where they were recorded.

\begin{table}[H]
    \centering
    \caption[SINS database recorded activities for each room]{SINS database recorded activities for each room \footnotemark}
	\label{tab:sins-database-recorded-activities}
    \begin{tabular}{l|l|c|c}
        \toprule
        \textbf{Room} & \textbf{Activity} & \textbf{Nr. ex.} & \textbf{duration (min.)} \\ 
        \midrule[1pt]
        \multirow{10}{*}{Living room} & Phone call & 22 & 8.17 $\pm$ 13.73 \\
        \cline{2-4}
        & Cooking & 19 & 16.62 $\pm$ 9.49 \\
        \cline{2-4}
        & Dish washing & 15 & 6.37 $\pm$ 1.49 \\
        \cline{2-4}
        & Eating & 19 & 7.78 $\pm$ 4.27 \\
        \cline{2-4}
        & Visit & 9 & 13.3 $\pm$ 12.11 \\
        \cline{2-4}
        & Watching TV & 13 & 155.38 $\pm$ 93.28 \\
        \cline{2-4}
        & Working & 15 & 31.24 $\pm$ 39.33 \\
        \cline{2-4}
        & Vacuum cleaning & 15 & 4.79 $\pm$ 2.14 \\
        \cline{2-4}
        & Other & 15 & 0.75 $\pm$ 0.95 \\
        \cline{2-4}
        & Absence & 15 & 66.37 $\pm$ 130.30 \\
        \midrule[1pt]
        \multirow{7}{*}{Bathroom} & Drying with towel & 10 & 1.67 $\pm$ 0.28 \\
        \cline{2-4}
        & Shaving & 13 & 1.91 $\pm$ 1.46 \\
        \cline{2-4}
        & Showering & 10 & 6.11 $\pm$ 2.38 \\
        \cline{2-4}
        & Tooth brushing & 19 & 1.41 $\pm$ 0.25 \\
        \cline{2-4}
        & Vacuum cleaning & 9 & 0.87 $\pm$ 0.59 \\
        \cline{2-4}
        & Other & 75 & 0.42 $\pm$ 0.4 \\
        \cline{2-4}
        & Absence & 35 & 248.56 $\pm$ 263.62 \\
        \midrule[1pt]
        \multirow{3}{*}{Hall} & Vacuum cleaning & 9 & 3.31 $\pm$ 1.11 \\
        \cline{2-4}
        & Other & 164 & 0.36 $\pm$ 0.22 \\
        \cline{2-4}
        & Absence & 175 & 50.17 $\pm$ 102.52 \\
        \midrule[1pt]
        \multirow{3}{*}{Toilet} & Toilet visit & 21 & 4.74 $\pm$ 3.24 \\
        \cline{2-4}
        & Vacuum cleaning & 7 & 0.53 $\pm$ 0.07 \\
        \cline{2-4}
        & Absence & 31 & 282.75 $\pm$ 263.19 \\
        \midrule[1pt]
        \multirow{5}{*}{Bedroom} & Dressing & 28 & 1.53 $\pm$ 1.10 \\
        \cline{2-4}
        & Sleeping & 7 & 348.43 $\pm$ 130.73 \\
        \cline{2-4}
        & Vacuum cleaning & 7 & 1.04 $\pm$ 0.27 \\
        \cline{2-4}
        & Other & 22 & 0.27 $\pm$ 0.23 \\
        \cline{2-4}
        & Absence & 22 & 122.28 $\pm$ 157.43 \\
        \bottomrule
    \end{tabular}
\end{table}
\footnotetext{\fullcite{dekkers_sins_2017}}

\clearpage
\landscapevalues

\section{Test concept}
\label{app:Test-Concept}
This section of the appendix contains detailed information about the manual tests conducted in the project to test each component thoroughly. For each manually conducted test case the steps, the status, the expected result and the actual result is given. The automated tests can be found in the \texttt{test} directory of the source code repository.

\begin{longtable}{p{.15\textwidth} | p{.40\textwidth} | p{.05\textwidth} | p{.15\textwidth} | p{.15\textwidth}}
        \caption{Test concept of the manual conducted tests}
	    \label{tab:Test-Concept} \\
        \toprule
        \textbf{Test case} & \textbf{Steps} & \textbf{Status} & \textbf{Expected result} & \textbf{Actual result} \\ 
    \midrule[1pt]
        Embedding models successfully train & 
        \begin{minipage}{3.8in}
        \vskip 4pt
        For each of the embedding models in the project:
        \begin{enumerate}
        \setlength\itemsep{0em}
        \item choose the embedding model to train
        \item train the model for ten epochs on the DCASE dataset
        \item check if the model started to converge during the training, by checking if the triplet loss value decreases
        \end{enumerate}
        \vskip 4pt
        \end{minipage} & \cellcolor{green!30!white}passed & models converge & all of the models converged during the training \\
    \hline
        Classifier models successfully train & 
        \begin{minipage}{3.8in}
        \vskip 4pt
        For each of the classifier models in the project:
        \begin{enumerate}
        \setlength\itemsep{0em}
        \item choose a trained embedding model to embed the segments into the embedding space
        \item choose the classifier model to train
        \item train the classifier for ten epochs on the DCASE dataset
        \item for each of the segments in the dataset, embed and then feed them to the classifier
        \item check if the model started to converge during the training, by checking if the sparse categorical cross-entropy loss decreases 
        \end{enumerate}
        \vskip 4pt
        \end{minipage} & \cellcolor{green!30!white}passed & classifiers converge & all of the classifiers converged during the training \\
    \hline
        Training process works & 
        \begin{minipage}{3.8in}
        \vskip 4pt
        \begin{enumerate}
        \setlength\itemsep{0em}
        \item choose a random embedding model, optimally not a very computing intensive one
        \item train the model for ten epochs on the DCASE dataset
        \item check if after ten epochs the training terminates
        \item check if there were no errors during the training, for example because of the input pipeline
        \item check if all of the metrics are correctly displayed on the Tensorboard
        \item check if all of the metrics seem reasonable, no extremely low or high values after a few epochs
        \item check if the model is successfully saved at the end of each epoch
        \end{enumerate}
        \vskip 4pt
        \end{minipage} & \cellcolor{green!30!white}passed & training process works & all of the checks were successfully and therefore the training process works \\
    \hline
        Training process stopped and started again & 
        \begin{minipage}{3.8in}
        \vskip 4pt
        \begin{enumerate}
        \setlength\itemsep{0em}
        \item choose a random embedding model, optimally not a very computing intensive one
        \item train the model for ten epochs on the DCASE dataset
        \item after five epochs, stop the training procedure
        \item start the training procedure again for five epochs with the same model, by changing the params in the \texttt{params.json} file
        \item check if the model was successfully loaded
        \item check if the model was trained from the fifth epoch again and not from scratch
        \item check if there were no errors during the training, for example because of the input pipeline
        \item check if the metrics show that the model was stopped during the training
        \item check if the model is successfully saved
        \end{enumerate}
        \vskip 4pt
        \end{minipage} & \cellcolor{green!30!white}passed & training process loads models correctly & all of the checks were successfully and therefore the training process for loading already trained models works, the metrics further show a continuous graph even if the training is stopped and than started again \\
    \bottomrule
\end{longtable}

\clearpage
\defaultpagestyle

\section{Study doc}
\label{app:Study-Doc}
This chapter contains all of the study docs from the conducted experiments, they include detailed information about each one of the experiments as well as the conclusion out of each.

\includepdf[scale=1.0, pages=1-]{study-doc/experiment_margin/Study_Doc_Margin.pdf}

\includepdf[scale=1.0, pages=1-]{study-doc/experiment_tile_size/Study_Doc_tile_size.pdf}

\includepdf[scale=1.0, pages=1-]{study-doc/experiment_embedding_size/Study_Doc_embedding_size.pdf}

\includepdf[scale=1.0, pages=1-]{study-doc/experiment_regularisation/Study_Doc_regularisation.pdf}

\includepdf[scale=1.0, pages=1-]{study-doc/experiment_feature/Study_Doc_feature_representation.pdf}

\section{Qualitative analysis}
\label{app:Qualitative-analysis}
This section shows the qualitative analysis conducted during the thesis to evaluate the music embedding space. The analysis was done in the form of an interview with Mr Emanuel Oehri as the expert, who is a DJ and provided the music dataset.

% Interview including answers
\includepdf[scale=1.0, pages=1-]{files/bachelor_thesis_interview.pdf}

% Web abstract
\includepdf[scale=0.75, pages=1,pagecommand={\section{Web abstract} \label{app:Web-abstract} \thispagestyle{empty}}, fitpaper=true]{files/bachelor_thesis_web_abstract.pdf}
\includepdf[scale=0.75, pages=2-,pagecommand={\thispagestyle{empty}}, fitpaper=true]{files/bachelor_thesis_web_abstract.pdf}

\section{Detailed hyperparameters of the models}
\label{app:detailed-hyperparameters}
This section contains detailed information about both of the final models trained in the thesis. \ref{tab:Hyperparameters-Detailed-DCASE} shows the detailed hyperparameters for the DCASE dataset, while \ref{tab:Hyperparameters-Detailed-Music} shows them for the music dataset.

\begin{table}[H]
    \centering
    \caption{Detailed hyperparameters of the optimal embedding architecture for the noise detection dataset}
	\label{tab:Hyperparameters-Detailed-DCASE}
    \begin{tabular}{l|l}
        \toprule
        \textbf{Hyperparameter} & \textbf{value} \\ 
        \midrule[1pt]
        Dataset & DCASE \\
        \hline
        Model & ResNet18 \\ 
        \hline
        Epochs & 110 \\ 
        \hline
        Batch size & 64 \\ 
        \hline
        Optimizer & Adam \\ 
        \hline
        Learning rate & 1e-5 \\
        \hline
        Margin & 1.0 \\
        \hline
        L2 regularisation amount & 0.01 \\
        \hline
        Embedding dimension & 256 \\
        \hline
        Prefetch batches & Autotune (-1) \\ 
        \hline
        Random selection buffer & 32 \\ 
        \hline
        Shuffle dataset & True \\
        \hline
        Random seed & 1234 \\
        \midrule[1pt]
        \multicolumn{2}{l}{\textit{Multi threading}} \\
        \midrule[1pt]
        Number of generators & 16 \\ 
        \hline
        Number of parallel calls & 16 \\
        \midrule[1pt]
        \multicolumn{2}{l}{\textit{Audio sample}} \\
        \midrule[1pt]
        Sample rate & 16000 \\ 
        \hline
        Sample size & 10s \\
        \hline
        Sample tile size & 5s \\
        \hline
        Sample tile range & 5s \\
        \hline
        Convert to mono & True \\
        \midrule[1pt]
        \multicolumn{2}{l}{\textit{Feature representation}} \\
        \midrule[1pt]
        Feature extractor & LogMelExtractor \\ 
        \hline
        Frame length & 480 \\
        \hline
        Frame step & 160 \\
        \hline
        FFT size & 1024 \\
        \hline
        Number of Mel bins & 128 \\
        \hline
        Number of MFCC bins & 13 \\
        \bottomrule
    \end{tabular}
\end{table}

\begin{table}[H]
    \centering
    \caption{Detailed hyperparameters of the optimal embedding architecture for the music dataset}
	\label{tab:Hyperparameters-Detailed-Music}
    \begin{tabular}{l|l}
        \toprule
        \textbf{Hyperparameter} & \textbf{value} \\ 
        \midrule[1pt]
        Dataset & MusicDataset \\
        \hline
        Model & ResNet18 \\ 
        \hline
        Epochs & 130 \\ 
        \hline
        Batch size & 32 \\ 
        \hline
        Optimizer & Adam \\ 
        \hline
        Learning rate & 1e-5 \\
        \hline
        Margin & 1.0 \\
        \hline
        L2 regularisation amount & 0.01 \\
        \hline
        Embedding dimension & 256 \\
        \hline
        Prefetch batches & Autotune (-1) \\ 
        \hline
        Random selection buffer & 16 \\ 
        \hline
        Shuffle dataset & True \\
        \hline
        Random seed & 1234 \\
        \midrule[1pt]
        \multicolumn{2}{l}{\textit{Multi threading}} \\
        \midrule[1pt]
        Number of generators & 16 \\ 
        \hline
        Number of parallel calls & 16 \\
        \midrule[1pt]
        \multicolumn{2}{l}{\textit{Audio sample}} \\
        \midrule[1pt]
        Sample rate & 44100 \\ 
        \hline
        Sample size & variable \\
        \hline
        Sample tile size & 10s \\
        \hline
        Sample tile range & 40s \\
        \hline
        Convert to mono & True \\
        \midrule[1pt]
        \multicolumn{2}{l}{\textit{Feature representation}} \\
        \midrule[1pt]
        Feature extractor & LogMelExtractor \\ 
        \hline
        Frame length & 480 \\
        \hline
        Frame step & 160 \\
        \hline
        FFT size & 1024 \\
        \hline
        Number of Mel bins & 128 \\
        \hline
        Number of MFCC bins & 13 \\
        \bottomrule
    \end{tabular}
\end{table}

\clearpage
\section{Usage instructions}
\label{app:Usage-Instructions}
This section describes how to use the source code to train an embedding space and to evaluate it using the classifier. The instructions can as well be found in the \texttt{README.md} file of the source repository. As a first step, the source repository has to be cloned.
\begin{code}[H]
\begin{minted}{shell}
git clone https://github.com/FabianGroeger96/deep-embedded-music.git
\end{minted}
\caption{Clone source repository}
\label{code:Clone-Repo}
\end{code}

\subsection{Requirements}
\begin{itemize}
\setlength\itemsep{0em}
\item
  \href{https://github.com/tensorflow/tensorflow}{TensorFlow 2.1}
\item
  \href{https://pandas.pydata.org/}{pandas}
\item
  \href{https://numpy.org/}{numpy}
\item
  \href{https://www.scipy.org/}{scipy}
\item
  \href{https://scikit-learn.org/stable/}{scikit-learn}
\item
  \href{https://github.com/librosa/librosa}{librosa}
\item
  \href{https://matplotlib.org/}{matplotlib}
\item
  \href{https://seaborn.pydata.org/}{seaborn}
\end{itemize}

\subsection{Train embedding space}
The following docker command can be executed using the \texttt{onstart.sh} script, to train the embedding space using triplet loss on a GPU. The container will start, and the training procedure begins right away.
\begin{code}[H]
\begin{minted}{shell}
docker run -it -v ${PWD}:/tf/ -w /tf --name deep-embedded-music \
--gpus all --privileged=true \
tensorflow/tensorflow:2.1.0-gpu-py3 /bin/bash ./onstart.sh
\end{minted}
\caption{Train embedding space using triplet loss}
\label{code:Train-triplet-loss}
\end{code}

\noindent
Take note of your Docker version with \texttt{docker -v}. Versions earlier than
19.03 require \texttt{nvidia-docker2} and the \texttt{--runtime=nvidia} flag. On versions
including and after 19.03, the use of the \texttt{nvidia-container-toolkit}
package and the \texttt{--gpus all} flag is needed.
\newline
\newline
The training process can be started manually without the \texttt{onstart.sh} script, by starting the docker container and then executing the python script \texttt{train\_triplet\_loss.py} manually.
\begin{code}[H]
\begin{minted}{shell}
cd deep-embedded-music
python -m src.train_triplet_loss
\end{minted}
\caption{Train embedding space using triplet loss manually}
\label{code:Train-triplet-loss-manually}
\end{code}
\noindent
If the tensorboard profiler wants to be used, by setting the parameter
\texttt{use\_profiler\ =\ 1}, the tensorflow nightly build
\texttt{tensorflow/tensorflow:nightly-gpu-py3} has to be used to start
the docker container as well as the option \texttt{--privileged=true}.
This is due to NVIDIA CUPTI libary API change:
\newline
\href{https://developer.nvidia.com/nvidia-development-tools-solutions-err-nvgpuctrperm-cupti}{https://developer.nvidia.com}

\subsection{Train classifier}
A separate python script has to be called named
\texttt{train\_classifier.py}, to train the classifier on top of the
embedding architecture. The \texttt{--model\_to\_load} argument
specifies which embedding model should be used to train the classifier,
the classifier will be trained using the same dataset.
\begin{code}[H]
\begin{minted}{shell}
cd deep-embedded-music
python -m src.train_classifier --model_to_load results/path_to_model/
\end{minted}
\caption{Train classifier on top of the embedding space}
\label{code:Train-classifier}
\end{code}

\subsection{Run tensorboard}
The tensorboard can be used, to examine the results from the trained
models. The tensorboard can be started by executing the docker command
below. The tensorboard is then available on \url{http://localhost:6006}.
The \texttt{--logdir} specifies the location of the results to
visualise.
\begin{code}[H]
\begin{minted}{shell}
docker run -it -p 6006:6006 --rm -v ${PWD}:/tf/ \
--name deep-embedded-music-tensorboard \
tensorflow/tensorflow:2.1.0-py3 \
tensorboard --bind_all --logdir tf/experiments/DCASE/results/
\end{minted}
\caption{Run tensorboard to visualise the experiments}
\label{code:Run-Tensorboard}
\end{code}

\clearpage
\section{Zip folder}
\label{app:Zip-folder}
Together with the thesis, a zip-folder was delivered. The components of it are illustrated in figure \ref{fig:Contents-Zip-Folder}. \texttt{audios\_from\_documentation} contains the audio files, referenced in the thesis, for example, in the examination of the DCASE embedding space. \texttt{dcase\_results} contains the fully trained models on the DCASE dataset, whereas \texttt{music\_results} contains the final models trained using the music dataset. \texttt{music\_best\_songs\_generated} contains some audio files, which were generated by the embedding space while examining it. \texttt{qualitative\_analysis} contains all of the resources for and from the qualitative analysis, such as the resources given to the expert, detailed interview guide and transcribed interview. Finally, \texttt{study\_docs} contains all of the study docs from different experiments.
\begin{figure}[ht]
    \dirtree{%
    .1 deep\_embedded\_music\_resources/.
    .2 audios\_from\_documentation/.
    .2 dcase\_results/.
    .2 music\_results/.
    .2 music\_best\_songs\_generated/.
    .2 qualitative\_analysis/.
    .2 study\_docs/.
    .2 deep\_embedded\_music\_code\_doc.pdf.
    }
\caption{Contents of the Zip-folder delivered with the thesis}
\label{fig:Contents-Zip-Folder}
\end{figure}

\clearpage
\landscapevalues

\section{Work journal}
\label{app:Work-Journal}
Within this section, the whole work journal of the thesis is shown. The journal is divided into subcategories, so that the \fullref{tab:Work-Journal} is more pleased to read. These categories show what exactly was done in the different project phases, which can be seen in the \fullref{fig:Project-Plan}. There is also one more category called \flqq General\frqq, where all the administrative tasks are shown, including the meetings with the advisor.

\begin{longtable}{| p{.10\textwidth} | p{.20\textwidth} | p{.50\textwidth} | p{.10\textwidth} |} 
	\caption{Work Journal}
	\label{tab:Work-Journal} \\
    \hline
    \textbf{Date} &
    \textbf{Task} &
    \textbf{Details} &
    \textbf{No. hours} \\
    \hline
    \multicolumn{4}{|l|}{\textbf{General}} \\
    \hline
    17.02.2020 & preparation for kick-off meeting & 
        \begin{minipage}{5in}
        \vskip 4pt
        \begin{itemize}
        \setlength\itemsep{0em}
        \item thesis description read carefully again
        \item questions of unclear matters
        \end{itemize}
        \vskip 4pt
        \end{minipage}
        & 1h  \\
    \hline
    17.02.2020 & kick-off meeting (number 1)& 
        \begin{minipage}{5in}
        \vskip 4pt
        \begin{itemize}
        \setlength\itemsep{0em}
        \item discussed different topics which are important for the implementation of the project
        \end{itemize}
        \vskip 4pt
        \end{minipage}
        & 1h 30min  \\
    \hline
    24.02.2020 & meeting (number 2) & 
        \begin{minipage}{5in}
        \vskip 4pt
        \begin{itemize}
        \setlength\itemsep{0em}
        \item preparation of the meeting
        \item meeting itself
        \end{itemize}
        \vskip 4pt
        \end{minipage}
        & 2h  \\
    \hline
    02.03.2020 & meeting (number 3) & 
        \begin{minipage}{5in}
        \vskip 4pt
        \begin{itemize}
        \setlength\itemsep{0em}
        \item preparation of the meeting
        \item meeting itself
        \end{itemize}
        \vskip 4pt
        \end{minipage}
        & 2h  \\
    \hline
    02.03.2020 & meeting GPU with Thomas Koller & 
        \begin{minipage}{5in}
        \vskip 4pt
        \begin{itemize}
        \setlength\itemsep{0em}
        \item preparation of the meeting
        \item meeting itself
        \end{itemize}
        \vskip 4pt
        \end{minipage}
        & 0.5h  \\
    \hline
    09.03.2020 & meeting (number 4) & 
        \begin{minipage}{5in}
        \vskip 4pt
        \begin{itemize}
        \setlength\itemsep{0em}
        \item preparation of the meeting
        \item meeting itself
        \end{itemize}
        \vskip 4pt
        \end{minipage}
        & 2h  \\
    \hline
    16.03.2020 & meeting (number 5) & 
        \begin{minipage}{5in}
        \vskip 4pt
        \begin{itemize}
        \setlength\itemsep{0em}
        \item preparation of the meeting
        \item meeting itself
        \end{itemize}
        \vskip 4pt
        \end{minipage}
        & 2h  \\
    \hline
    23.03.2020 & meeting (number 6) & 
        \begin{minipage}{5in}
        \vskip 4pt
        \begin{itemize}
        \setlength\itemsep{0em}
        \item preparation of the meeting
        \item meeting itself
        \end{itemize}
        \vskip 4pt
        \end{minipage}
        & 2h  \\
    \hline
    30.03.2020 & meeting (number 7) & 
        \begin{minipage}{5in}
        \vskip 4pt
        \begin{itemize}
        \setlength\itemsep{0em}
        \item preparation of the meeting
        \item meeting itself
        \end{itemize}
        \vskip 4pt
        \end{minipage}
        & 2h  \\
    \hline
    06.04.2020 & meeting (number 8) & 
        \begin{minipage}{5in}
        \vskip 4pt
        \begin{itemize}
        \setlength\itemsep{0em}
        \item preparation of the meeting
        \item meeting itself
        \end{itemize}
        \vskip 4pt
        \end{minipage}
        & 2h  \\
    \hline
    15.04.2020 & meeting (number 9) & 
        \begin{minipage}{5in}
        \vskip 4pt
        \begin{itemize}
        \setlength\itemsep{0em}
        \item preparation of the meeting
        \item meeting itself
        \end{itemize}
        \vskip 4pt
        \end{minipage}
        & 2h  \\
    \hline
    20.04.2020 & meeting (number 10) & 
        \begin{minipage}{5in}
        \vskip 4pt
        \begin{itemize}
        \setlength\itemsep{0em}
        \item preparation of the meeting
        \item meeting itself
        \end{itemize}
        \vskip 4pt
        \end{minipage}
        & 2h  \\
    \hline
    21.04.2020 & interim presentation & 
        \begin{minipage}{5in}
        \vskip 4pt
        \begin{itemize}
        \setlength\itemsep{0em}
        \item preparation
        \item presentation
        \item discussion
        \end{itemize}
        \vskip 4pt
        \end{minipage}
        & 5h  \\
    \hline
    27.04.2020 & meeting (number 11) & 
        \begin{minipage}{5in}
        \vskip 4pt
        \begin{itemize}
        \setlength\itemsep{0em}
        \item preparation of the meeting
        \item meeting itself
        \end{itemize}
        \vskip 4pt
        \end{minipage}
        & 2h  \\
    \hline
    04.05.2020 & meeting (number 12) & 
        \begin{minipage}{5in}
        \vskip 4pt
        \begin{itemize}
        \setlength\itemsep{0em}
        \item preparation of the meeting
        \item meeting itself
        \end{itemize}
        \vskip 4pt
        \end{minipage}
        & 2h  \\
    \hline
    11.05.2020 & meeting (number 13) & 
        \begin{minipage}{5in}
        \vskip 4pt
        \begin{itemize}
        \setlength\itemsep{0em}
        \item preparation of the meeting
        \item meeting itself
        \end{itemize}
        \vskip 4pt
        \end{minipage}
        & 2h  \\
    \hline
    19.05.2020 & meeting (number 14) & 
        \begin{minipage}{5in}
        \vskip 4pt
        \begin{itemize}
        \setlength\itemsep{0em}
        \item preparation of the meeting
        \item meeting itself
        \end{itemize}
        \vskip 4pt
        \end{minipage}
        & 2h  \\
    \hline
    25.05.2020 & meeting (number 15) & 
        \begin{minipage}{5in}
        \vskip 4pt
        \begin{itemize}
        \setlength\itemsep{0em}
        \item preparation of the meeting
        \item meeting itself
        \end{itemize}
        \vskip 4pt
        \end{minipage}
        & 2h  \\
    \hline
    28.05.2020 & interview Emanuel Oehri & 
        \begin{minipage}{5in}
        \vskip 4pt
        \begin{itemize}
        \setlength\itemsep{0em}
        \item preparation of interview guide
        \item preparation of resources
        \item interview itself
        \end{itemize}
        \vskip 4pt
        \end{minipage}
        & 6h  \\
    \hline
    \multicolumn{4}{|l|}{\textbf{Documentation}} \\
    \hline
    17.02.2020 & documentation setup & 
        \begin{minipage}{5in}
        \vskip 4pt
        \begin{itemize}
        \setlength\itemsep{0em}
        \item project setup
        \item latex document setup
        \item German transcribed to English
        \item built default documentation structure
        \end{itemize}
        \vskip 4pt
        \end{minipage}
        & 2h  \\
    \hline
    19.02.2020 & dataset documented & 
        \begin{minipage}{5in}
        \vskip 4pt
        \begin{itemize}
        \setlength\itemsep{0em}
        \item SINS database
        \item DCASE task dataset
        \end{itemize}
        \vskip 4pt
        \end{minipage}
        & 3h  \\
    \hline
    24.02.2020 & dataset section finished & 
        \begin{minipage}{5in}
        \vskip 4pt
        \begin{itemize}
        \setlength\itemsep{0em}
        \item created table of the recorded activities in the \gls{SINS} database
        \item reread section and corrected
        \end{itemize}
        \vskip 4pt
        \end{minipage}
        & 1h  \\
    \hline
    02.03.2020 & triplet Loss section finished & 
        \begin{minipage}{5in}
        \vskip 4pt
        \begin{itemize}
        \setlength\itemsep{0em}
        \item documented Triplet Loss in related work
        \item reread section and corrected
        \end{itemize}
        \vskip 4pt
        \end{minipage}
        & 2h  \\
    \hline
    03.03.2020 & Tile2Vec section finished & 
        \begin{minipage}{5in}
        \vskip 4pt
        \begin{itemize}
        \setlength\itemsep{0em}
        \item documented Tile2Vec in related work
        \item reread section and corrected
        \item corrected equations
        \end{itemize}
        \vskip 4pt
        \end{minipage}
        & 2h  \\
    \hline
    05.03.2020 & started with intro to neural networks & 
        \begin{minipage}{5in}
        \vskip 4pt
        \begin{itemize}
        \setlength\itemsep{0em}
        \item researched neural networks
        \item documented neural networks
        \end{itemize}
        \vskip 4pt
        \end{minipage}
        & 4h  \\
    \hline
    06.03.2020 & finished with intro to neural networks & 
        \begin{minipage}{5in}
        \vskip 4pt
        \begin{itemize}
        \setlength\itemsep{0em}
        \item documented neural networks
        \end{itemize}
        \vskip 4pt
        \end{minipage}
        & 4h  \\
    \hline
    05.03.2020 & started with intro to convolutional neural networks & 
        \begin{minipage}{5in}
        \vskip 4pt
        \begin{itemize}
        \setlength\itemsep{0em}
        \item researched convolutional neural networks
        \item documented convolutional neural networks
        \end{itemize}
        \vskip 4pt
        \end{minipage}
        & 2h  \\
    \hline
    09.03.2020 & finished intro to convolutional neural networks & 
        \begin{minipage}{5in}
        \vskip 4pt
        \begin{itemize}
        \setlength\itemsep{0em}
        \item documented convolutional neural networks
        \end{itemize}
        \vskip 4pt
        \end{minipage}
        & 3h  \\
    \hline
    09.03.2020 & documented various parts & 
        \begin{minipage}{5in}
        \vskip 4pt
        \begin{itemize}
        \setlength\itemsep{0em}
        \item documented project plan and milestones
        \item documented introduction
        \item documented appendix
        \item documented into to clustering
        \end{itemize}
        \vskip 4pt
        \end{minipage}
        & 3h  \\
    \hline
    11.03.2020 & document changes applied from meeting & 
        \begin{minipage}{5in}
        \vskip 4pt
        \begin{itemize}
        \setlength\itemsep{0em}
        \item moving Dataset to related work
        \item changed denotation of loss function
        \item changed \gls{NN} section
        \end{itemize}
        \vskip 4pt
        \end{minipage}
        & 1h  \\
    \hline
    11.03.2020 & finished intro to gated recurrent unit & 
        \begin{minipage}{5in}
        \vskip 4pt
        \begin{itemize}
        \setlength\itemsep{0em}
        \item documented gated recurrent unit
        \end{itemize}
        \vskip 4pt
        \end{minipage}
        & 2h  \\
    \hline
    13.03.2020 & finished status in relation to project & 
        \begin{minipage}{5in}
        \vskip 4pt
        \begin{itemize}
        \setlength\itemsep{0em}
        \item researched state-of-the-art in audio deep learning
        \item documented status in relation to project
        \end{itemize}
        \vskip 4pt
        \end{minipage}
        & 4h  \\
    \hline
    15.03.2020 & finished related work & 
        \begin{minipage}{5in}
        \vskip 4pt
        \begin{itemize}
        \setlength\itemsep{0em}
        \item researched dilated convolution
        \item documented dilated convolution
        \end{itemize}
        \vskip 4pt
        \end{minipage}
        & 2h  \\
    \hline
    15.03.2020 & milestone review & 
        \begin{minipage}{5in}
        \vskip 4pt
        \begin{itemize}
        \setlength\itemsep{0em}
        \item M1 milestone review
        \end{itemize}
        \vskip 4pt
        \end{minipage}
        & 1h  \\
    \hline
    27.03.2020 & started with changes from Daniel & 
        \begin{minipage}{5in}
        \vskip 4pt
        \begin{itemize}
        \setlength\itemsep{0em}
        \item related work changes
        \end{itemize}
        \vskip 4pt
        \end{minipage}
        & 2h  \\
    \hline
    28.03.2020 & started with chapter 3, ideas and concepts & 
        \begin{minipage}{5in}
        \vskip 4pt
        \begin{itemize}
        \setlength\itemsep{0em}
        \item data preprocessing
        \item feature extraction
        \item data augmentation
        \end{itemize}
        \vskip 4pt
        \end{minipage}
        & 3h  \\
    \hline
    29.03.2020 & milestone review & 
        \begin{minipage}{5in}
        \vskip 4pt
        \begin{itemize}
        \setlength\itemsep{0em}
        \item M2 milestone review
        \end{itemize}
        \vskip 4pt
        \end{minipage}
        & 1h  \\
    \hline
    06.04.2020 & finished with changes from Daniel & 
        \begin{minipage}{5in}
        \vskip 4pt
        \begin{itemize}
        \setlength\itemsep{0em}
        \item related Work changes
        \end{itemize}
        \vskip 4pt
        \end{minipage}
        & 1h  \\
    \hline
    06.04.2020 & worked on chapter 3, ideas and concepts & 
        \begin{minipage}{5in}
        \vskip 4pt
        \begin{itemize}
        \setlength\itemsep{0em}
        \item input pipeline
        \item triplet selection
        \item refactoring of chapter
        \end{itemize}
        \vskip 4pt
        \end{minipage}
        & 3h 30min  \\
    \hline
    12.04.2020 & milestone review & 
        \begin{minipage}{5in}
        \vskip 4pt
        \begin{itemize}
        \setlength\itemsep{0em}
        \item M3 milestone review
        \end{itemize}
        \vskip 4pt
        \end{minipage}
        & 1h  \\
    \hline
    13.04.2020 & worked on chapter 3, ideas and concepts & 
        \begin{minipage}{5in}
        \vskip 4pt
        \begin{itemize}
        \setlength\itemsep{0em}
        \item rewrote most of the sections
        \item finished draft of chapter 3
        \end{itemize}
        \vskip 4pt
        \end{minipage}
        & 3h \\
    \hline
    13.04.2020 & worked on chapter 4, method &
        \begin{minipage}{5in}
        \vskip 4pt
        \begin{itemize}
        \setlength\itemsep{0em}
        \item updated project plan
        \item started with documentation of procedure model
        \end{itemize}
        \vskip 4pt
        \end{minipage}
        & 1h \\
    \hline
    15.04.2020 & worked on interim presentation &
        \begin{minipage}{5in}
        \vskip 4pt
        \begin{itemize}
        \setlength\itemsep{0em}
        \item created slides
        \item reviewed slides
        \end{itemize}
        \vskip 4pt
        \end{minipage}
        & 4h \\
    \hline
    16.04.2020 & worked on interim presentation &
        \begin{minipage}{5in}
        \vskip 4pt
        \begin{itemize}
        \setlength\itemsep{0em}
        \item practised presentation
        \end{itemize}
        \vskip 4pt
        \end{minipage}
        & 4h \\
    \hline
    20.04.2020 & worked on interim presentation &
        \begin{minipage}{5in}
        \vskip 4pt
        \begin{itemize}
        \setlength\itemsep{0em}
        \item practised presentation
        \end{itemize}
        \vskip 4pt
        \end{minipage}
        & 2h \\
    \hline
    24.04.2020 & worked on chapter 4, method &
        \begin{minipage}{5in}
        \vskip 4pt
        \begin{itemize}
        \setlength\itemsep{0em}
        \item project structure
        \item resources
        \item evaluation
        \end{itemize}
        \vskip 4pt
        \end{minipage}
        & 6h \\
    \hline
    25.04.2020 & worked on chapter 5, realisation &
        \begin{minipage}{5in}
        \vskip 4pt
        \begin{itemize}
        \setlength\itemsep{0em}
        \item overall structure
        \item started with UML diagrams
        \end{itemize}
        \vskip 4pt
        \end{minipage}
        & 3h \\
    \hline
    30.04.2020 & thesis changes of chapter 1, 3, 4 and 5 &
        \begin{minipage}{5in}
        \vskip 4pt
        \begin{itemize}
        \setlength\itemsep{0em}
        \item thesis corrections
        \end{itemize}
        \vskip 4pt
        \end{minipage}
        & 4h \\
    \hline
    03.05.2020 & milestone review & 
        \begin{minipage}{5in}
        \vskip 4pt
        \begin{itemize}
        \setlength\itemsep{0em}
        \item M5 milestone review
        \end{itemize}
        \vskip 4pt
        \end{minipage}
        & 1h  \\
    \hline
    06.05.2020 & worked on chapter 5, realisation &
        \begin{minipage}{5in}
        \vskip 4pt
        \begin{itemize}
        \setlength\itemsep{0em}
        \item documented realisation
        \item created UML diagrams
        \end{itemize}
        \vskip 4pt
        \end{minipage}
        & 4h \\
    \hline
    08.05.2020 & worked on chapter 5, realisation &
        \begin{minipage}{5in}
        \vskip 4pt
        \begin{itemize}
        \setlength\itemsep{0em}
        \item finished with realisation
        \end{itemize}
        \vskip 4pt
        \end{minipage}
        & 6h \\
    \hline
    23.05.2020 & worked on chapter 6, results &
        \begin{minipage}{5in}
        \vskip 4pt
        \begin{itemize}
        \setlength\itemsep{0em}
        \item started with results from DCASE challenge dataset
        \end{itemize}
        \vskip 4pt
        \end{minipage}
        & 4h \\
    \hline
    25.05.2020 & worked on chapter 4, method &
        \begin{minipage}{5in}
        \vskip 4pt
        \begin{itemize}
        \setlength\itemsep{0em}
        \item added detailed information about work packages, artefacts and phases
        \item added risk analysis
        \end{itemize}
        \vskip 4pt
        \end{minipage}
        & 3h \\
    \hline
    30.05.2020 & worked on chapter 6, results &
        \begin{minipage}{5in}
        \vskip 4pt
        \begin{itemize}
        \setlength\itemsep{0em}
        \item results from DCASE challenge dataset
        \item results from music dataset
        \end{itemize}
        \vskip 4pt
        \end{minipage}
        & 6h \\
    \hline
    31.05.2020 & worked on chapter 6, results &
        \begin{minipage}{5in}
        \vskip 4pt
        \begin{itemize}
        \setlength\itemsep{0em}
        \item finished with chapter
        \end{itemize}
        \vskip 4pt
        \end{minipage}
        & 3h \\
    \hline
    31.05.2020 & worked on chapter 7, conclusion &
        \begin{minipage}{5in}
        \vskip 4pt
        \begin{itemize}
        \setlength\itemsep{0em}
        \item project conclusion
        \item discussion
        \item outlook
        \end{itemize}
        \vskip 4pt
        \end{minipage}
        & 5h \\
    \hline
    01.06.2020 & thesis corrections &
        \begin{minipage}{5in}
        \vskip 4pt
        \begin{itemize}
        \setlength\itemsep{0em}
        \item corrections
        \end{itemize}
        \vskip 4pt
        \end{minipage}
        & 4h \\
    \hline
    02.06.2020 & thesis corrections &
        \begin{minipage}{5in}
        \vskip 4pt
        \begin{itemize}
        \setlength\itemsep{0em}
        \item corrections
        \end{itemize}
        \vskip 4pt
        \end{minipage}
        & 5h \\
    \hline
    03.06.2020 & thesis corrections &
        \begin{minipage}{5in}
        \vskip 4pt
        \begin{itemize}
        \setlength\itemsep{0em}
        \item corrections
        \end{itemize}
        \vskip 4pt
        \end{minipage}
        & 4h \\
    \hline
    04.06.2020 & thesis corrections &
        \begin{minipage}{5in}
        \vskip 4pt
        \begin{itemize}
        \setlength\itemsep{0em}
        \item corrections
        \end{itemize}
        \vskip 4pt
        \end{minipage}
        & 7h \\
    \hline
    \multicolumn{4}{|l|}{\textbf{Research}} \\
    \hline
    17.02.2020 & research dataset & 
        \begin{minipage}{5in}
        \vskip 4pt
        \begin{itemize}
        \setlength\itemsep{0em}
        \item finished with DCASE - challenge description
        \item started with SINS database paper
        \end{itemize}
        \vskip 4pt
        \end{minipage}
        & 2h 30min  \\
    \hline
    21.02.2020 & audio processing research & 
        \begin{minipage}{5in}
        \vskip 4pt
        \begin{itemize}
        \setlength\itemsep{0em}
        \item \gls{FT}
        \item \gls{FFT}
        \item \gls{DFT}
        \item Spectrogram
        \item started with related work documentation
        \end{itemize}
        \vskip 4pt
        \end{minipage}
        & 5h  \\
    \hline
    26.02.2020 & audio processing research & 
        \begin{minipage}{5in}
        \vskip 4pt
        \begin{itemize}
        \setlength\itemsep{0em}
        \item \gls{MFCC}
        \item documenting research in related work
        \end{itemize}
        \vskip 4pt
        \end{minipage}
        & 3h  \\
    \hline
    27.02.2020 & audio processing research & 
        \begin{minipage}{5in}
        \vskip 4pt
        \begin{itemize}
        \setlength\itemsep{0em}
        \item \gls{MFCC}
        \item finished documenting research in related work
        \end{itemize}
        \vskip 4pt
        \end{minipage}
        & 4h  \\
    \hline
    28.02.2020 & triplet loss research & 
        \begin{minipage}{5in}
        \vskip 4pt
        \begin{itemize}
        \setlength\itemsep{0em}
        \item read various paper on triplet loss
        \end{itemize}
        \vskip 4pt
        \end{minipage}
        & 3h  \\
    \hline
    01.03.2020 & triplet loss research & 
        \begin{minipage}{5in}
        \vskip 4pt
        \begin{itemize}
        \setlength\itemsep{0em}
        \item current research on triplet loss with audio
        \end{itemize}
        \vskip 4pt
        \end{minipage}
        & 2h  \\
    \hline
    02.03.2020 & Tile2Vec research & 
        \begin{minipage}{5in}
        \vskip 4pt
        \begin{itemize}
        \setlength\itemsep{0em}
        \item read various paper on tile2vec
        \item started documenting tile2vec in related work
        \end{itemize}
        \vskip 4pt
        \end{minipage}
        & 5h  \\
    \hline
    31.05.2020 & documented other relevant work & 
        \begin{minipage}{5in}
        \vskip 4pt
        \begin{itemize}
        \setlength\itemsep{0em}
        \item summarised other relevant work for the thesis
        \end{itemize}
        \vskip 4pt
        \end{minipage}
        & 2h  \\
    \hline
    \multicolumn{4}{|l|}{\textbf{Realisation}} \\
    \hline
    14.03.2020 & started with project setup and input pipeline & 
        \begin{minipage}{5in}
        \vskip 4pt
        \begin{itemize}
        \setlength\itemsep{0em}
        \item set up the GitHub project
        \item started with input pipeline
        \item generator of triplets
        \end{itemize}
        \vskip 4pt
        \end{minipage}
        & 8h  \\
    \hline
    15.03.2020 & started with input pipeline & 
        \begin{minipage}{5in}
        \vskip 4pt
        \begin{itemize}
        \setlength\itemsep{0em}
        \item first test cases for input pipeline
        \item implemented first idea of similarity measure between audios
        \end{itemize}
        \vskip 4pt
        \end{minipage}
        & 2h  \\
    \hline
    16.03.2020 & input pipeline finished & 
        \begin{minipage}{5in}
        \vskip 4pt
        \begin{itemize}
        \setlength\itemsep{0em}
        \item finished input pipeline
        \item finished unit test for pipeline
        \end{itemize}
        \vskip 4pt
        \end{minipage}
        & 7h  \\
    \hline
    18.03.2020 & project setup & 
        \begin{minipage}{5in}
        \vskip 4pt
        \begin{itemize}
        \setlength\itemsep{0em}
        \item finished project structure
        \end{itemize}
        \vskip 4pt
        \end{minipage}
        & 2h  \\
    \hline
    18.03.2020 & implementation & 
        \begin{minipage}{5in}
        \vskip 4pt
        \begin{itemize}
        \setlength\itemsep{0em}
        \item triplet loss implemented
        \item dense model implemented
        \item started with projector visualisation
        \end{itemize}
        \vskip 4pt
        \end{minipage}
        & 5h  \\
    \hline
    20.03.2020 & implementation & 
        \begin{minipage}{5in}
        \vskip 4pt
        \begin{itemize}
        \setlength\itemsep{0em}
        \item train workflow implemented
        \end{itemize}
        \vskip 4pt
        \end{minipage}
        & 5h  \\
    \hline
    23.03.2020 & implementation & 
        \begin{minipage}{5in}
        \vskip 4pt
        \begin{itemize}
        \setlength\itemsep{0em}
        \item tested models and workflow on GPU
        \item convolutional model implemented
        \end{itemize}
        \vskip 4pt
        \end{minipage}
        & 5h  \\
    \hline
    25.03.2020 & implementation & 
        \begin{minipage}{5in}
        \vskip 4pt
        \begin{itemize}
        \setlength\itemsep{0em}
        \item convolutional model edited (1D and 2D)
        \item training workflow edited
        \end{itemize}
        \vskip 4pt
        \end{minipage}
        & 4h  \\
    \hline
    27.03.2020 & implementation & 
        \begin{minipage}{5in}
        \vskip 4pt
        \begin{itemize}
        \setlength\itemsep{0em}
        \item factories implemented
        \item metrics implemented
        \end{itemize}
        \vskip 4pt
        \end{minipage}
        & 5h  \\
    \hline
    31.03.2020 & implementation & 
        \begin{minipage}{5in}
        \vskip 4pt
        \begin{itemize}
        \setlength\itemsep{0em}
        \item GRU Model implemented
        \item refactoring
        \end{itemize}
        \vskip 4pt
        \end{minipage}
        & 4h  \\
    \hline
    03.04.2020 & implementation & 
        \begin{minipage}{5in}
        \vskip 4pt
        \begin{itemize}
        \setlength\itemsep{0em}
        \item music dataset pipeline
        \item unit tests finished
        \item documentation of code
        \item worked on classifier
        \end{itemize}
        \vskip 4pt
        \end{minipage}
        & 6h  \\
    \hline
    04.04.2020 & implementation & 
        \begin{minipage}{5in}
        \vskip 4pt
        \begin{itemize}
        \setlength\itemsep{0em}
        \item worked on evaluation workflow
        \end{itemize}
        \vskip 4pt
        \end{minipage}
        & 3h  \\
    \hline
    07.04.2020 & implementation & 
        \begin{minipage}{5in}
        \vskip 4pt
        \begin{itemize}
        \setlength\itemsep{0em}
        \item input pipeline multiprocessing
        \end{itemize}
        \vskip 4pt
        \end{minipage}
        & 5h  \\
    \hline
    08.04.2020 & implementation & 
        \begin{minipage}{5in}
        \vskip 4pt
        \begin{itemize}
        \setlength\itemsep{0em}
        \item changed triplet selection of datasets
        \end{itemize}
        \vskip 4pt
        \end{minipage}
        & 6h  \\
    \hline
    12.04.2020 & implementation & 
        \begin{minipage}{5in}
        \vskip 4pt
        \begin{itemize}
        \setlength\itemsep{0em}
        \item finished evaluation workflow
        \item added new metrics to classifier
        \end{itemize}
        \vskip 4pt
        \end{minipage}
        & 4h  \\
    \hline
    14.04.2020 & implementation & 
        \begin{minipage}{5in}
        \vskip 4pt
        \begin{itemize}
        \setlength\itemsep{0em}
        \item input pipeline speedup
        \end{itemize}
        \vskip 4pt
        \end{minipage}
        & 3h  \\
    \hline
    15.04.2020 & implementation & 
        \begin{minipage}{5in}
        \vskip 4pt
        \begin{itemize}
        \setlength\itemsep{0em}
        \item added ResNet architecture
        \end{itemize}
        \vskip 4pt
        \end{minipage}
        & 4h  \\
    \hline
    22.04.2020 & implementation & 
        \begin{minipage}{5in}
        \vskip 4pt
        \begin{itemize}
        \setlength\itemsep{0em}
        \item bug fixes
        \item added zero filtering triplet loss strategy
        \end{itemize}
        \vskip 4pt
        \end{minipage}
        & 4h  \\
    \hline
    28.04.2020 & implementation & 
        \begin{minipage}{5in}
        \vskip 4pt
        \begin{itemize}
        \setlength\itemsep{0em}
        \item regularisation added to models
        \end{itemize}
        \vskip 4pt
        \end{minipage}
        & 4h  \\
    \hline
    30.04.2020 & implementation & 
        \begin{minipage}{5in}
        \vskip 4pt
        \begin{itemize}
        \setlength\itemsep{0em}
        \item bug fixes classifier
        \item bug fixes metrics
        \end{itemize}
        \vskip 4pt
        \end{minipage}
        & 3h  \\
    \hline
    05.05.2020 & implementation & 
        \begin{minipage}{5in}
        \vskip 4pt
        \begin{itemize}
        \setlength\itemsep{0em}
        \item added logistic classifier
        \end{itemize}
        \vskip 4pt
        \end{minipage}
        & 1h  \\
    \hline
    08.05.2020 & implementation & 
        \begin{minipage}{5in}
        \vskip 4pt
        \begin{itemize}
        \setlength\itemsep{0em}
        \item big refactoring
        \item code documentation
        \end{itemize}
        \vskip 4pt
        \end{minipage}
        & 4h  \\
    \hline
    \multicolumn{4}{|l|}{\textbf{Experiments}} \\
    \hline
    17.04.2020 until 22.04.2020 & experiment margin & 
        \begin{minipage}{5in}
        \vskip 4pt
        \begin{itemize}
        \setlength\itemsep{0em}
        \item started experiment
        \item evaluated results
        \item created study doc template
        \item wrote study doc
        \end{itemize}
        \vskip 4pt
        \end{minipage}
        & 8h  \\
    \hline
    22.04.2020 until 28.04.2020 & experiment feature & 
        \begin{minipage}{5in}
        \vskip 4pt
        \begin{itemize}
        \setlength\itemsep{0em}
        \item started experiment
        \item evaluated results
        \item wrote study doc
        \end{itemize}
        \vskip 4pt
        \end{minipage}
        & 6h  \\
    \hline
    30.04.2020 until 05.05.2020 & experiment tile size & 
        \begin{minipage}{5in}
        \vskip 4pt
        \begin{itemize}
        \setlength\itemsep{0em}
        \item started experiment
        \item evaluated results
        \item wrote study doc
        \end{itemize}
        \vskip 4pt
        \end{minipage}
        & 6h  \\
    \hline
    06.05.2020 until 11.05.2020 & experiment embedding size & 
        \begin{minipage}{5in}
        \vskip 4pt
        \begin{itemize}
        \setlength\itemsep{0em}
        \item started experiment
        \item evaluated results
        \item wrote study doc
        \end{itemize}
        \vskip 4pt
        \end{minipage}
        & 8h  \\
    \hline
    11.05.2020 until 13.05.2020 & experiment regularisation factor & 
        \begin{minipage}{5in}
        \vskip 4pt
        \begin{itemize}
        \setlength\itemsep{0em}
        \item started experiment
        \item evaluated results
        \item wrote study doc
        \end{itemize}
        \vskip 4pt
        \end{minipage}
        & 5h  \\
    \hline
    15.05.2020 until 20.05.2020 & experiment embedding size larger & 
        \begin{minipage}{5in}
        \vskip 4pt
        \begin{itemize}
        \setlength\itemsep{0em}
        \item started experiment
        \item evaluated results
        \item wrote study doc
        \end{itemize}
        \vskip 4pt
        \end{minipage}
        & 6h  \\
    \hline
    20.05.2020 until 24.05.2020 & final model for DCASE dataset trained & 
        \begin{minipage}{5in}
        \vskip 4pt
        \begin{itemize}
        \setlength\itemsep{0em}
        \item final model trained
        \item evaluated results
        \end{itemize}
        \vskip 4pt
        \end{minipage}
        & 3h  \\
    \hline
    23.05.2020 until 30.05.2020 & final model for music dataset trained & 
        \begin{minipage}{5in}
        \vskip 4pt
        \begin{itemize}
        \setlength\itemsep{0em}
        \item trained model using the same hyperparameters on the music dataset
        \item evaluated results
        \end{itemize}
        \vskip 4pt
        \end{minipage}
        & 5h  \\
    \hline
\end{longtable}

\clearpage
\defaultpagestyle

\section{Task definition bachelor thesis}
\label{app:Task-Definition}
Within this chapter the final task definition, which was submitted and accepted by the Transfer Office of the Lucerne University of Applied Sciences and Arts on the 25.02.2020, is attached.

\includepdf[scale=1.0, pages=1-]{files/Aufgabenstellung_DeepEmbeddedMusic.pdf}


\end{document}
