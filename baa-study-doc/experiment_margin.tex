\documentclass[twocolumn]{article}
\setlength{\columnsep}{20pt}
%\usepackage{url}
%\usepackage{algorithmic}
\usepackage[a4paper]{geometry}
\usepackage{datetime}
\usepackage[font=small,labelfont=it]{caption}
\usepackage{graphicx}
\usepackage{mathpazo} % use palatino
\usepackage[scaled]{helvet} % helvetica
\usepackage{microtype}
\usepackage{amsmath}
\usepackage{subfigure}

% Letterspacing macros
\newcommand{\spacecaps}[1]{\textls[200]{\MakeUppercase{#1}}}
\newcommand{\spacesc}[1]{\textls[50]{\textsc{\MakeLowercase{#1}}}}

\title{\spacecaps{Lab report: Lab 0 }\\ \normalsize \spacesc{TNM079, Modeling and Animation} }

\author{Your Name\\yourliuid@student.liu.se}
%\date{\today\\\currenttime}
\date{\today}

\begin{document}
\maketitle

\begin{abstract}
%
This template outlines the different sections that are required for the reports for the lab course in TNM079 at Link\"oping University. You are encouraged to use the source for this template to produce your own lab reports. Each section title is followed by a short description and possible requirements. First out is the abstract. It should give a rough overview of the report, including purpose, key insights and major conclusions. It can additionally include a brief description of the methods used. The abstract should consist of about 100 words. The upper limit would be around 200 words. Try write one or two sentences about \textit{what}, \textit{how} and \textit{why} additionally you should give first results. Keep it compressed and always treat is as a summary of your report.

Remember that a good lab report does more than presenting results; it demonstrates the writer's comprehension of the concepts and insights behind the methods.

The report should be self-contained, which means that it should not be necessary to read other sources such as the papers that you list in your reference list.
%
\end{abstract}


\section{Introduction}
%
The introduction is there to set the stage and define the subject of the report. It should also answer the following questions: \emph{Why is the lab performed? What is the specific purpose of this lab? Why is this lab important?}. It carries on where the abstract stopped but focuses on the topic itself. No results nor any background.

\section{Background}
%
The section \textit{Background} should provide a description of every task you implemented. It should take up about 1 A4 page if you aim at a grade 3, 1.5 A4 pages for a 4 and about 2 A4 pages if you aim at a 5 (excluding Figures). You give in this section a more detailed description of what you have implemented in this lab. You can use 2-3 figures to support your description. If you find other work which can be referenced, this would be a good place to put it.

\section{Tasks}
%
This section should describe what work was done and how, specifically how you solved the different tasks. For this type of lab report it can be beneficial to put each assignment in its own subsection.

\subsection{``Your Subtask''}

When working with equations, remember that it is not enough to simply state that you used this or that equation to solve the problem. You should also (briefly) describe what each equation accomplishes, what its parts are and \emph{what it means in the particular setting of this lab}. You can also mention in which file the implementations were made.

\newpage
For this assignment, the following equation regarding academic output was used:

\begin{align}
    r &= e \cdot \Delta_t \label{eq:abc}
\end{align}

In Equation~\ref{eq:abc}, $e$ represents the total body of work (effort) while $\Delta_t$ represents time. In the context of this report, the result $r$ represents the quality of the report.

\section{Results}

The point of the results section is to provide detailed information regarding the output of the lab. This section is often dominated by tables and screenshots. Make sure that each screenshot has a purpose and the purpose is communicated to the reader! You should show at least 2 figures for grade 3, 3 for grade 4 and 4 for grade 5. The \emph{Results} and \emph{Conclusion} sections together should result in 1 A4 page if you aim for a 3, 1.5 pages if you aim for a 4 and 2 A4 pages for a 5 (including Figures).

Your grade will not benefit from putting unnecessary screenshots just to fill the blank paper! Also remember that every figure needs a meaningful caption as in Figure~\ref{fig:demo}. Note how captions can be used to direct the reader towards the interesting bits of an image. The image should be understood by only reading the caption. Also, there is a package in latex that allows you to put images / figures / screenshots next to each other, its called \emph{subfigure} and used in the example in Figure~\ref{fig:demo}.

Much of the validation of the course labs is done through visual output. Screenshots are encouraged, but make sure they provide some useful insights. A two- or three-way comparison (perhaps with some particularly interesting part zoomed in) is typically a great way to present a single task. In case the tasks main output are numbers, make sure you include them, as this is typically a great way to compare, for example, performance, deviation from analytical solutions, e.g. surface area, volume, problem sizes and so on. As an example: \emph{In table 1.1 we show the area of different meshes using our implementation.}

\begin{figure}[t]
\centering
    \subfigure[]{\includegraphics[width=.3\linewidth]{fig/flat.png}\label{fig:demo-standard}}
    \subfigure[]{\includegraphics[width=.3\linewidth]{fig/smooth.png}\label{fig:demo-fancy}}
    \caption{Result from task X showing \subref{fig:demo-standard} standard rendering compared to fancy rendering in \subref{fig:demo-fancy}. Notice the improved smoothness of the fancy rendering which was achieved through the use of Equation~\ref{eq:abc} and \emph{knowledge}. Shading is typically a trade-off between computation time and visual quality.}
    \label{fig:demo}
\end{figure}


\section{Conclusion}
For higher grades it is important that you not only solve the assignments but also discuss your findings. This will show your level of comprehension of the lab and is an important ingredient when assessing the lab report.

When discussing your results try to compare against known constants and expressions. For example the volume of a sphere is well known and can be compared against: \emph{We found the volume of the approximated sphere mesh (a) to be within the xy error margin. With higher tessellation the error is reduced by xy.}

If something is known or has been measured, it goes in \emph{Results}. If something is situational/conditional or requires reasoning or discussion, it goes in \emph{Conclusions}.

Your reports \textbf{total number} of pages should be equal to the grade you are aiming for.

\section{Lab partner and grade}
For the records, give the name of your lab partner, if you have one. This section should also contain the grade you are aiming for.

\nocite{*}
\bibliographystyle{plain}
\bibliography{references}
\end{document}